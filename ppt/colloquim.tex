%----------------------------------------------------------------------------------------
%	PACKAGES AND THEMES
%----------------------------------------------------------------------------------------

\documentclass{beamer}

\mode<presentation> {

% The Beamer class comes with a number of default slide themes
% which change the colors and layouts of slides. Below this is a list
% of all the themes, uncomment each in turn to see what they look like.

%\usetheme{default}
%\usetheme{AnnArbor}
%\usetheme{Antibes}
%\usetheme{Bergen}
%\usetheme{Berkeley}
%\usetheme{Berlin}
%\usetheme{Boadilla}
%\usetheme{CambridgeUS}
%\usetheme{Copenhagen}
%\usetheme{Darmstadt}
%\usetheme{Dresden}
%\usetheme{Frankfurt}
%\usetheme{Goettingen}
%\usetheme{Hannover}
%\usetheme{Ilmenau}
%\usetheme{JuanLesPins}
%\usetheme{Luebeck}
\usetheme{Madrid}
%\usetheme{Malmoe}
%\usetheme{Marburg}
%\usetheme{Montpellier}
%\usetheme{PaloAlto}
%\usetheme{Pittsburgh}
%\usetheme{Rochester}
%\usetheme{Singapore}
%\usetheme{Szeged}
%\usetheme{Warsaw}

% As well as themes, the Beamer class has a number of color themes
% for any slide theme. Uncomment each of these in turn to see how it
% changes the colors of your current slide theme.

%\usecolortheme{albatross}
%\usecolortheme{beaver}
%\usecolortheme{beetle}
%\usecolortheme{crane}
%\usecolortheme{dolphin}
%\usecolortheme{dove}
%\usecolortheme{fly}
%\usecolortheme{lily}
%\usecolortheme{orchid}
%\usecolortheme{rose}
%\usecolortheme{seagull}
%\usecolortheme{seahorse}
%\usecolortheme{whale}
%\usecolortheme{wolverine}

%\setbeamertemplate{footline} % To remove the footer line in all slides uncomment this line
%\setbeamertemplate{footline}[page number] % To replace the footer line in all slides with a simple slide count uncomment this line

%\setbeamertemplate{navigation symbols}{} % To remove the navigation symbols from the bottom of all slides uncomment this line
}

\usepackage{graphicx} % Allows including images
\usepackage{booktabs} % Allows the use of \toprule, \midrule and \bottomrule in tables

\makeatletter
\setbeamertemplate{footline}
{
	\leavevmode%
	\hbox{%
		\begin{beamercolorbox}[wd=.333333\paperwidth,ht=2.25ex,dp=1ex,center]{author in head/foot}%
			\usebeamerfont{author in head/foot}\insertshorttitle %\insertauthor
		\end{beamercolorbox}%
		\begin{beamercolorbox}[wd=.333333\paperwidth,ht=2.25ex,dp=1ex,center]{title in head/foot}%
			\usebeamerfont{title in head/foot}\insertsection
		\end{beamercolorbox}%
		\begin{beamercolorbox}[wd=.333333\paperwidth,ht=2.25ex,dp=1ex,right]{date in head/foot}%
			\usebeamerfont{date in head/foot}\insertshortdate{}\hspace*{2em}
			\insertframenumber{} / \inserttotalframenumber\hspace*{2ex} 
		\end{beamercolorbox}}%
		\vskip0pt%
	}
\makeatother
	
%----------------------------------------------------------------------------------------
%	TITLE PAGE
%----------------------------------------------------------------------------------------

\title[HAShCache]{HAShCache : Heterogeneity Aware Shared DRAMCache
	for Integrated Heterogeneous Systems} % The short title appears at the bottom of every slide, the full title is only on the title page

\author{Adarsh Patil} % Your name
\institute[CSA, IISc] % Your institution as it will appear on the bottom of every slide, may be shorthand to save space
{
Department of Computer Science and Automation \\ Indian Institute of Science \\ % Your institution for the title page
\medskip
\textit{Advisor: Prof. R. Govindarajan} % Advisor
}
\date{\today} % Date, can be changed to a custom date

\begin{document}

\begin{frame}
\titlepage % Print the title page as the first slide
\centering \tiny{This work has been submitted to 50th Internation Symposium on Microarchitecture (MICRO) 2017}
\end{frame}

\begin{frame}
\frametitle{Overview} % Table of contents slide, comment this block out to remove it
\tableofcontents % Throughout your presentation, if you choose to use \section{} and \subsection{} commands, these will automatically be printed on this slide as an overview of your presentation
\end{frame}

%----------------------------------------------------------------------------------------
%	PRESENTATION SLIDES
%----------------------------------------------------------------------------------------

%------------------------------------------------
\section{Introduction}
%------------------------------------------------
\subsection{Intergrated Heterogenous Systems (IHS)} 

%------------------------------------------------
\begin{frame}
\frametitle{Introduction}
\begin{table}[]
\small
\centering
\begin{tabular}{@{}lll@{}}
	\toprule Criteria
	& \textbf{CPU}                                                                                     & \textbf{GPU}                                                                                  \\ \midrule
	\begin{tabular}[c]{@{}l@{}}number of \\ processing units\end{tabular} & 4-16 CPUs                                                                                        & 2048 processing units                                                                         \\ \midrule
	ILP                                                                   & 128 entry Re-order Buffer                                                                        & \begin{tabular}[c]{@{}l@{}}max 64 warps resident\\ (each warp has 32 threads)\end{tabular}    \\ \midrule
	\begin{tabular}[c]{@{}l@{}}Last Level SRAM\\ Cache\end{tabular}       & 1-4 MB                                                                                           & 700 KB - 1 MB                                                                                 \\ \midrule
	Memory                                                                & \begin{tabular}[c]{@{}l@{}}Large Main Memory\\ $\sim$ 10s - 100s GB\\ Allows Paging\end{tabular} & \begin{tabular}[c]{@{}l@{}}Limited Memory\\ $\sim$ 8 - 16 GB\\ No Paging support\end{tabular} \\ \midrule
	\begin{tabular}[c]{@{}l@{}}Programming\\ Paradigm\end{tabular}        & \begin{tabular}[c]{@{}l@{}}serial / multi-threaded /\\ SIMD\end{tabular}                         & SIMT, explicitly parallel                                                                     \\ \midrule
	Programming                                                           & Easy - Medium                                                                                    & Difficult                                                                                     \\ \midrule
	Power                                                                 & 1.5 GFLOPS/Watt                                                                                  & 5 GFLOPS / Watt                                                                               \\ \bottomrule
\end{tabular}
\label{cpu-vs-gpu}
\end{table}
\end{frame}

\note[itemize]{
	\item power 2012 K20 NVIDIA vs 2012 Sandybridge E5-2670
	\item energy efficiency of GPUs comes from simpler cores (small out of orders, no Branch pred)
	\item maybe add that GPU is streaming type of core
}
%------------------------------------------------

\begin{frame}
\frametitle{Integrated Heterogenous Systems (IHS) Architecture}
Latency-oriented CPU cores + Throughput-oriented GPGPU SMs \textbf{on-chip} \\
\begin{itemize}
	\item Simplifies Programming - Shared Virtual Memory, pointer sharing
	\item Allows GPGPUs to operate on data sets larger memory size %using OS paging routines
	\item Reduces overall energy consumption
	\item Share resources - NoC, caches, memory controllers, DRAMs
	\item e.g. AMD APUs, Intel Iris, NVIDIA Denver
\end{itemize}
\begin{figure}[!htb]
	\centering
	\def\svgwidth{0.6\columnwidth}
	\input{arch.pdf_tex}
	%\caption{Architecture of an Integrated Heterogeneous System}
	\label{fig:hsa-arch}
\end{figure}
\end{frame}

\note[itemize]{
\item 99\% of Intel and 67\% of AMD non-server processors contain integrated graphics [APU 2013 summit]
\item most smartphones and tablet processors ship with heterogenous processors.
}
%------------------------------------------------
\subsection{3D Die-stacked DRAM} 

\begin{frame}
\frametitle{3D Die-stacked DRAM}
\begin{itemize}
\item DRAM layers placed closer to processing cores using 2.5D interposer and/or 3D TSV (through-silicon-vias)
\item Large capacity DRAM chips of the order of 100s of MBs to GBs
\item Large DRAM Bandwidth (beneficial for GPGPUs)
\item Better interconnect slightly lowers access latency (beneficial for CPUs)
\item Our proposal uses this capacity as a hardware managed cache
\item e.g. HBM (AMD/Hynix), HMC (Intel/Micron)
\end{itemize}
\begin{figure}[!htb]
	\centering
	\def\svgwidth{0.7\columnwidth}
	\input{stackedDRAM.pdf_tex}
	%\caption{Architecture of an Integrated Heterogeneous System}
	\label{fig:stacked-dram}
\end{figure}
\end{frame}

%------------------------------------------------
\section{Motivation}
%------------------------------------------------

\begin{frame}
\frametitle{Motivation}
(a) CPU Performance in IHS with Naive DRAMCache
    \begin{figure}
    	\includegraphics[scale=1]{graphs/motivation-cpu}
    \end{figure}
\begin{itemize}
\item CPU performance improves by 42\% over baseline IHS system without DRAMCache
\item Homogenous CPU archieves 372\% improvement with a DRAMCache
\item Performance gap of almost 260\%  compared to the ideal HoA CPU
\end{itemize} 
\end{frame} 
  
\begin{frame}
\frametitle{Motivation}    
(b) GPU Perfomance in IHS with Naive DRAMCache    
    \begin{figure}    	
    	\includegraphics[scale=1]{graphs/motivation-gpu}
    \end{figure}
\begin{itemize}
	\item GPU performance improves by 24\% over baseline IHS system with without DRAMCache
	\item Homogenous GPU archieves 35\% improvement with a DRAMCache
	\item Performance gap within 10\%  of its ideal HoA GPU
\end{itemize}    
\end{frame}

\begin{frame}
\frametitle{Motivation}
Cause for CPU Performance gap
	\begin{figure}
		\includegraphics[scale=1]{graphs/motivation-cpu-cache}
	\end{figure}
\begin{itemize}
	\item CPU Hit rates in DRAMCache marginally affected (about 4\%)
	\item Co-running with GPU increases the avg mem access latency by 213\%
\end{itemize}	
\end{frame}

%------------------------------------------------
\section{HAShCache Design}
%------------------------------------------------

\begin{frame}
	\frametitle{HAShCache Design}
	\begin{block}{Metadata overhead}
		\small
		Caching at 128 byte granularity - Experimental results with 512 byte block under performs 128 bytes blocks by (12.2\%, 10.5\%) for (CPU,GPU)\\
		Tags stored in DRAM alongside data and streamed out in multiple bursts (TADs)
	\end{block}
	\begin{block}{Set Associativity}
		\small
		Direct mapped cache organization
	\end{block}
	\begin{block}{Reducing Miss Penalty}
		\small
		Cache Hit/miss Predictor based on MAP-I prediction for CPU \\
		No such mechanism for GPU
	\end{block}
	\begin{block}{Addressing Scheme}
		\small
		Addressed with RBH addressing scheme (Row-Rank-Bank-Column-Channel) \\ 
		Experimental results show that BLP addressing scheme  (RoCoRaBaCh) performs worse by (3\%, 1\%) for (CPU,GPU)
	\end{block}	
\end{frame}

\note[itemize]{
	\item one needs to keep in mind several factors when designing a DRAM based cache as compared to a conventional cache
	\item several organizations have been proposed with various trade offs for DRAMCache in the cache parameters
	\item we re-examine these design decisions in the context of IHS architectures
}


%------------------------------------------------
\section{HAShCache Mechanisms}
%------------------------------------------------
\subsection{Heterogeneity aware DRAMCache scheduler - PrIS}


\begin{frame}
\frametitle{Heterogeneity aware DRAMCache scheduler - PrIS}
\begin{columns}[c] % The "c" option specifies centered vertical alignment while the "t" option is used for top vertical alignment

\column{.45\textwidth} % Left column and width
\textbf{Heading}
\begin{enumerate}
\item Statement
\item Explanation
\item Example
\end{enumerate}

\column{.5\textwidth} % Right column and width
Lorem ipsum dolor sit amet, consectetur adipiscing elit. Integer lectus nisl, ultricies in feugiat rutrum, porttitor sit amet augue. Aliquam ut tortor mauris. Sed volutpat ante purus, quis accumsan dolor.

\end{columns}
\end{frame}

%------------------------------------------------

\subsection{Heterogeneity aware Temporal bypass - ByE}
\begin{frame}
\frametitle{Heterogeneity aware Temporal bypass - ByE}

\end{frame}
%------------------------------------------------

\subsection{}
\begin{frame}
\frametitle{Heterogeneity aware Spatial Occupancy Control - Chaining}
	
\end{frame}

%------------------------------------------------
\section{Evaluation Methodology}
\begin{frame}
	\frametitle{Experimental Setup}
\begin{columns}[c]
\column{.75\textwidth} % Right column and width
	
	\begin{table}[h]
		\footnotesize
		\centering
		\begin{tabular}{{@{}ll@{}}}
			\toprule
			CPU Core 	& five 4-wide OoO x86 cores @2.5 GHz \\
			\midrule
			CPU Caches 	& 32KB 8-way split I/D private L1 Cache, \\ 
			& 1MB 8-way shared split L2 Cache, 128B lines \\
			\midrule
			GPU Core 	& eight Fermi SMs@700MHz, 2x2 GTO warp sched\\
			& 64K registers, 96KB scratch memory \\
			\midrule
			GPU Caches 	& 64KB 4-way private L1 cache,\\ 
			& 512KB, 16-way assoc L2 Coherent Cache \\
			\midrule
			Stacked     & 2 vaults, HMC\_2500\_x64, 2KB Page \\
			DRAM		& $t_{CL}$-$t_{RCD}$-$t_{RP}$-$t_{RAS}$=9.9ns-10.2ns-7.7ns-21.6ns\\
			& 8 layers/vault, 4 banks/layer\\
			& 64 byte burst size, Peak bandwidth 40GB/s \\
			%& Refresh related: $t_{REFI}$=3.9us $t_{RFC}$=59us \\
			\midrule
			Off-chip 	& 2 channels, DDR3\_1600\_x64, 1KB page \\
			DRAM		& $t_{CL}$-$t_{RCD}$-$t_{RP}$-$t_{RAS}$=13.75ns-13.75ns-13.75ns-35ns\\
			& 1 rank/channel, 8 banks/rank\\
			& 64 byte burst size, Peak bandwidth 25GB/s \\
			%& Refresh related: $t_{REFI}$=7.8us $t_{RFC}$=260us \\
			\bottomrule
		\end{tabular}
		\caption{Configuration of the simulated system}
		\label{configuration}
	\end{table}
		
	\column{.2\textwidth} % Left column and width
	\textbf{Simulator}
	\begin{itemize}
		\item gem5-gpu
		\item SE mode
	\end{itemize}	
\end{columns}
\end{frame}

%------------------------------------------------

\begin{frame}
\frametitle{Methodology}
    Medium to high memory intensive \emph{Multi-programmed SPEC 2006} combinations on CPU coupled with a \emph{Rodinia no-copy} GPU benchmark
	\begin{table}[h!]
		\footnotesize
		\centering
		\begin{tabular}{{|l|l|l|}}
			\hline
			\textbf{Name} & \textbf{Multi-program SPEC2006} & \textbf{Rodinia}\\
			\hline
			Qg1& cactusADM;gcc;bzip2;sphinx3 & needle\\
			\hline
			Qg2 & astar;mcf;gcc;bzip2 & needle\\
			\hline
			Qg3 & gcc;libquantum;leslie3d;bwaves & needle\\
			\hline
			Qg4 & astar;soplex;cactusADM;libquantum & k-means\\
			\hline
			Qg5 & milc;mcf;libquantum;bzip2 & k-means\\
			\hline
			Qg6 & bzip2;gobmk;hmmer;sphinx3 & k-means\\
			\hline
			Qg7 & soplex;milc;cactusADM;libquantum & gaussian\\
			\hline
			Qg8 & milc;libquantum;gobmk;leslie3d & gaussian\\
			\hline
			Qg9 & astar;milc;gcc;leslie3d & hotspot\\
			\hline
			Qg10 & gcc;gobmk;leslie3d;sphinx3 & hotspot\\
			\hline
			Qg11 & astar;cactusADM;libquantum;sphinx3 & srad\\
			\hline
			Qg12 & astar;mcf;gobmk;sphinx3 & streamcluster\\
			\hline
			Qg13 & astar;cactusADM;libquantum;sphinx3 & lud\\
			\hline
		\end{tabular}
		\caption{Workloads}
		\label{workloads}
	\end{table}
\end{frame}



%------------------------------------------------

\begin{frame}[fragile] % Need to use the fragile option when verbatim is used in the slide
\frametitle{Verbatim}
\begin{example}[Theorem Slide Code]
\begin{verbatim}
\begin{frame}
\frametitle{Theorem}
\begin{theorem}[Mass--energy equivalence]
$E = mc^2$
\end{theorem}
\end{frame}\end{verbatim}
\end{example}
\end{frame}

%------------------------------------------------

\begin{frame}
\frametitle{Figure}
Uncomment the code on this slide to include your own image from the same directory as the template .TeX file.
%\begin{figure}
%\includegraphics[width=0.8\linewidth]{test}
%\end{figure}
\end{frame}

%------------------------------------------------

\begin{frame}[fragile] % Need to use the fragile option when verbatim is used in the slide
\frametitle{Citation}
An example of the \verb|\cite| command to cite within the presentation:\\~

This statement requires citation \cite{p1}.
\end{frame}

%------------------------------------------------

\begin{frame}
\frametitle{References}
\footnotesize{
\begin{thebibliography}{99} % Beamer does not support BibTeX so references must be inserted manually as below
\bibitem[Smith, 2012]{p1} John Smith (2012)
\newblock Title of the publication
\newblock \emph{Journal Name} 12(3), 45 -- 678.
\end{thebibliography}
}
\end{frame}

%------------------------------------------------

\begin{frame}
\Huge{\centerline{The End}}
\end{frame}

%----------------------------------------------------------------------------------------

\end{document}