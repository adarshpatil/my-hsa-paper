\section{Conclusion} \label{conclusion}
In this work, we presented a case for improvement of performance for CPU and GPU with the introduction of a DRAMCache in an IHS architecture with CPU and GPU cores. We quantify the effects of interference due to co-running on each processor and show that in IHS processors CPU performance is adversely impacted compared to the GPU. 
Second, we propose an effective DRAMCache design and show that addition of such a cache can be a very good fit for each processor type to improve system performance. For this we adapt an aggressive direct mapped Alloy cache with careful consideration of the IHS workloads.
Third, we improve system performance using a heterogeneity aware scheme for this DRAMCache. \cachename improves performance using heterogeneity aware DRAMCache scheduling, spatial and temporal techniques and apply these suitably for the first time in an IHS architecture. Our \cachename achieves significant improvement of about 200\% in overall system performance on an average over a baseline system with no DRAMCache and 41\% over a carefully designed but heterogeneity naive DRAMCache. This work thus shows that there are significant benefits of using a stacked DRAMCache for IHS processors, far exceeding the usefulness of the such devices in homogeneous GPUs and multi-core CPU systems.