\chapter{\cachename\ Results} \label{chap:results}
In this chapter, we evaluate the performance of the proposed \cachename\ mechanisms discussed in Chapter \ref{chap:hashcache} and show that \cachename\ is able to improve the performance of IHS. Further, we show that \cachename\ solution is robust and scales well by performing sensitivity study for the various components in Section \ref{sensitivity-results}. Lastly, we compare our schemes quantitatively with a couple of state-of-the-art techniques in Section \ref{comparison}

\section{\cachename\ Mechanisms}
In this section, we present the performance of \cachename\ mechanisms (\prioname, \bypassname\ and \chaining) and some combinations of them. We examine the impact of these mechanisms on CPUs, GPUs and the performance of the IHS system as a whole. Throughout, we use the experimental methodology as described in Section \ref{simulation-methodology}. 
\par In the following subsections we discuss the results in detail for each of the \cachename\ schemes in detail. Figure \ref{results-speedup} (a) and (b) presents the performance improvement in terms of Harmonic Mean of IPC, for CPU and GPU respectively, normalized to the baseline without a DRAMCache. We report the performance improvement due to the introduction of DRAMCache (Naive),  and then the different HASHCache mechanisms  and their combinations. In all cases the CPU applications are co-run with the GPU applications. For comparison reasons, we also show the performance of HoA CPU --- CPU applications (multi-programmed workloads) when run without the GPU application --- and  HoA GPU --- GPU application run without CPU applications.

\begin{figure}[htb]
	\centering
	\includegraphics{graphs/results-cpu}
	\includegraphics{graphs/results-gpu}
	\caption{Speedups obtained by adding a stacked DRAMCache for (a) CPU (b) GPU}
	\label{results-speedup}
\end{figure}

\subsection{PrIS DRAMCache scheduling}
Prioritizing CPU requests with our \prioname\ scheduler at the DRAMCache controller leads to considerable performance benefits for the CPU applications. By using \prioname, the average access latency of the DRAMCache for CPU requests reduces by an average of 15.3\% and upto 48.9\% over a naive DRAMCache (see Figure \ref{results-pris}). Hence, \prioname\ is able to improve the performance of the CPU by an average of 35\% over a naive DRAMCache. However, on the flip side giving aggressive priority to all CPU requests reduces the performance of GPU by 10\% despite the GPU being able to tolerate larger memory access latencies. For some of the benchmarks like Qg3, Qg10, Qg11 and Qg12 the high priority given to CPU requests by \prioname\ impacts the GPU, causing the GPU performance to reduce marginally below the baseline IHS architecture without a DRAMCache. Note however that, in these workloads, the introduction of DRAM cache (naive) itself improves performance only marginally. Our mechanisms (\bypassname and \chaining) further aim to reduce this performance drop for the GPU by ensuring (a) there are fewer CPU requests in the DRAMCache queues (b) GPU requests, despite being de-prioritized at DRAMCache, have a better hit rate in the DRAMCache and avoid accessing the off-chip DRAM.

\begin{figure}[htb]
	\centering
	\includegraphics{graphs/results-pris}
	\caption{DRAMCache Memory Memory Access Latency with }
	\label{results-pris}
\end{figure}

\subsection{ByE for Temporal bypass}
\bypassname\ attempts to improve performance by directing some CPU requests to be served from the off-chip DRAM, thus achieving improved resource utilization and bandwidth balance in the process. \bypassname\ alone achieves 12\% improvement in CPU performance and a 3\% improvement in GPU performance over a naive DRAMCache.
\par The CPU performance improvements are primarily due to bypassed requests facing reduced queuing delays at DRAM. Figure \ref{results-bloom} shows the percent reduction in total memory access latency on primary y-axis (left) for CPU read requests achieved by \bypassname\ over an already aggressive naive DRAMCache which employs a hit/miss predictor for CPU requests. The total memory access latency for CPU read requests reduces by an average of 28\%. The already high hit rates for GPU in the DRAMCache coupled with the no-fill policy for bypassed CPU requests ensures fewer GPU requests at the off-chip DRAM which leads to lesser congestion. 
\par Figure \ref{results-bloom} also shows the percentage of incoming read requests bypassed by \bypassname, on the secondary y axis. 
\bypassname\ is able to bypass on an average about 37\% of incoming read requests. On an average 23\% read requests are to dirty lines in the cache which cannot be bypassed.  The remaining 40\% are false positives in our Bloom filter implementation which could have bypassed the DRAMCache. We discuss more on a sensitive study of Bloom filter later in this section. Further, the reduced set contention and lesser number of CPU requests in DRAMCache queues reduces congestion for GPU, which in turn leads to small performance benefits for the GPU as well improving the Harmonic Mean of GPU IPC by 3\% over naive DRAMCache.
\par Another point to be noted here is that, unlike \prioname\ or \prioname+\bypassname, \bypassname\ is able to improve not just CPU application performance but also achieves improved performance for GPU applications over a naive DRAMCache. \bypassname\ achieves this by being able reduce the interference of CPU and GPU requests at DRAMCache. Thus allowing the GPU to better utilize the DRAMCache bandwidth and capacity while the redirected CPU requests are served out of off-chip DRAM memory.

\begin{figure}[!htb]
    \centering
    \includegraphics[scale=1.05]{graphs/results-bypass}
    \caption{Total MemAccLat reduction with \bypassname\ and \% of bypassed read requests for CPU requests}
    \label{results-bloom}
\end{figure}

\par Combining \prioname\ with \bypassname\ allows for the non-bypassed CPU requests at the DRAMCache to be served with a higher priority and hence reducing the queuing delays. \bypassname+\prioname\ performs better than just \prioname\ by 10\% for CPU and 7\% for GPU. Overall \bypassname+\prioname\ achieves 48.5\% improvement in CPU performance over a naive DRAMCache while degrading GPU performance by just 3\%.

\par The somewhat high false positive ratio in our bypass mechanism is due to the large number of dirty blocks in the DRAMCache and the relatively small size of the Bloom filter.  We also experimented with 3 hash-functions while also optimally increasing the array capacity to 312KB (20\% larger) to reduce aliasing and increase the efficacy of bypass. The performance of CPU and GPU using the larger bloom filter (in terms of Harmonic Mean of IPC) normalized to the original bloom filter size is shown in Figure \ref{large-bloom}. We observe that the CPU performance improves only by 2.1\% on an average while the GPU remains largely unaffected. Hence, we retain the original bloom filter (256KB) with 2 hash functions for our \bypassname\ implementation.

\begin{figure}[htb]
	\centering
	\includegraphics{graphs/results-large-bloom}
	\caption{Performance with a larger bloom filter}
	\label{large-bloom}
\end{figure}

\subsection{Chaining for Spatial Occupancy Control}
As discussed in Section \ref{mechanism-chaining}, \chaining\ mechanism improves hit rates for GPU while guaranteeing some occupancy for the CPU in the DRAMCache. We empirically determine a suitable low occupancy threshold of the CPU (\textit{$l_{cpu}$}) to be 20\% for all our workloads. Chaining alone performs no better than a naive cache as the queuing latencies overwhelm any improvements in hit rates. However, when chaining is coupled with \prioname, the increased hit rates reduces the performance drop caused by \prioname\ for the GPU from 10\% to 5.2\% (i.e., 4.8\% performance improvement over \prioname). For the CPU, guaranteed occupancy in the DRAMCache and the secondary effect of lower congestion at off-chip DRAM allows CPU requests to be serviced with lower delays. This improves performance of CPU by 7\%, over only \prioname.
\par Overall \chaining+\prioname\ improves CPU performance by 44.7\% while degrading GPU performance by merely 7\% over a naive DRAMCache.
% %This improves overall system throughput by a significant 37.7\% over a un-optimized DRAMCache.

\subsection{Summary of \cachename\ Mechanisms Performance}
We now holistically examine the performance improvements due to the introduction of our \cachename\ mechanisms, in  CPU and GPU cores together. From Figure \ref{results-speedup} we observe that, adding a naive DRAMCache can achieve an average of 42\% and 24\% improvement in CPU and GPU cores respectively. Whereas, \cachename\ achieves significant speedups of (205\%,17.5\%) and (211\%,20.4\%) for (CPU,GPU) using Heterogeneity-aware mechanisms of \bypassname\ and \chaining\ respectively. This comes within 16\% and 13\% of the ideal no interference performance for the GPU (final gray bar in Figure \ref{results-speedup}) and within 81\% and 76\% of the ideal no interference performance for the CPU (final gray bar in Figure \ref{results-speedup}(b)) for each of the schemes. Further, for memory intensive combination of CPU and GPU workloads like Qg7 and Qg8 which see significant degradation in performance of both processors due to interference, adding a DRAMCache can improve performance upto 430\% for CPU and 48\% for GPU over a baseline system with no stacked DRAMCache. In comparison, the naive DRAMCache implementation only brings 55\% and 56\% improvement in the CPU and GPU performance respectively.

\subsection{System Performance with \cachename}
\begin{figure}[!htb]
	\centering
	\includegraphics{graphs/results-system}
	\includegraphics{graphs/results-weighted-speedup}
	\caption{(a)Harmonic Mean (b)Weighted Speedup of IHS with \cachename}
	\label{results-system}
\end{figure}
\par We now examine aggregate IHS performance using system metrics discussed in section \ref{perf-metrics}. Firstly, Figure \ref{results-system}(a) plots the performance improvement as a Harmonic mean of IPCs of IHS IPC (combined CPU cores and GPU CUs), normalized to our baseline IHS architecture without a DRAMCache. Overall with simple heterogeneity aware management of the stacked DRAMCache, IHS systems can achieve on an average 200\% (upto 400\%) performance improvement while a naive DRAMCache is able to achieve just 41.8\% improvement.
\par Secondly, Figure \ref{results-system}(b) plots the weighted speedup normalized to the baseline of IHS without DRAMCache. Adding a DRAMCache to IHS processors naively achieves an improvement of merely 3\%. In fact for some workloads like Qg3 and Qg11 adding a DRAMCache without careful considerations can lead to negative performance impact. However, our heterogeneity-aware mechanisms are able to achieve on an average of 16\% and 15\%  improvement in performance over a IHS architecture without a DRAMCache. This improvement corresponds to a 12.9\% and 11.5\% improvement for each of the schemes over a carefully designed but heterogeneity unaware DRAMCache for the IHS processors.


\section{Sensitivity Study} \label{sensitivity-results}

Next we study the sensitivity of \cachename\ mechanisms with regard to a few memory system parameters.

\subsection{Larger Capacity DRAMCache}

We conduct a sensitivity study for \cachename\ mechanisms with a larger 128MB stacked DRAMCache. Figure \ref{results-128m} presents the performance of a naive DRAMCache and our mechanisms \bypassname+\prioname\ and \chaining+\prioname\ for the (a)CPU and (b)GPU in terms of H-Mean of IPC normalized to an IHS with no DRAMCache. We observe that the larger naive DRAMCache is able to achieve 54.5\% and 29.8\% improvement for CPU and GPU processors. \cachename\ mechanisms continue to provide an average improvement of 46\% and 39\% for the CPU, on an average, above a heterogeneity unaware DRAMCache and 226\% and 215\% over the baseline IHS. 
In the larger DRAMCache, Chaining+PrIS results in 15.8\% improvement over baseline (10\% decrease over naive), we
expect the scheme to catch up the lost performance (over naive) in benchmarks with larger GPU footprint.

\begin{figure}[!htb]
	\centering
	\includegraphics{graphs/results-128M-cpu}
	\includegraphics{graphs/results-128M-gpu}
	\caption{Sensitivity Study with 128MB DRAMCache (a)CPU (b)GPU}
	\label{results-128m}
\end{figure}


\subsection{Off-chip DRAM with same latency as DRAMCache}

\begin{figure}[!htb]
	\centering
	\includegraphics{graphs/results-ddr3-2133-cpu}
	\includegraphics{graphs/results-ddr3-2133-gpu}
	\caption{Sensitivity Study with DDR3-2133 off-chip DRAM (a)CPU (b)GPU}
	\label{results-2xbw}
\end{figure}

Although we expect the smaller sizes of stacked DRAMs to have lower latency than off-chip DRAMs, commercial stacked DRAM like those in Intel's Knights Landing \cite{xeonphi} have access latencies similar to off-chip DRAMs. Hence, we scale up the off-chip DRAM to DDR3-2133 like device with latencies (13.09ns-13.09ns-13.09ns-33ns) for $t_{CL}$-$t_{RCD}$-$t_{RP}$-$t_{RAS}$. As before, Figure \ref{results-2xbw} presents the performance (H-Mean of IPC) of (a)CPU (b)GPU with a naive DRAMCache and \cachename\ normalized to a IHS system with no DRAMCache. For \prioname+\bypassname, CPU performance impoves by 48\% while reducing merely 3\% of GPU performance. For \prioname+\chaining, CPU performs 47\% better than a naive DRAMCache while decreasing GPU performance by 4.4\%.

\subsection{Higher Bandwidth DRAMCache}

\begin{figure}[!htb]
	\centering
	\includegraphics{graphs/results-2xbw-cpu}
	\includegraphics{graphs/results-2xbw-gpu}
	\caption{Sensitivity Study with 2x Bandwidth for DRAMCache (a)CPU (b)GPU}
	\label{results-ddr3-2133}
\end{figure}

We carry out experiments for \cachename\ with a stacked DRAMCache of 80GB/s. We achieve this by doubling the burst length of the stacked DRAMCache devices. Again, Figure \ref{results-ddr3-2133}  presents the performance of a naive DRAMCache and \cachename\ for (a)CPU and (b)GPU. We observe that our mechanisms scale with increased DRAMCache bandwidth and \bypassname+\prioname\ performs 51\% better and \chaining+\prioname\ performs 47.3\% better than a naive DRAMCache for CPU. GPU performs also improves by 1.9\% and 2.4\% over a naive DRAMCache. Notably for GPU benchmarks like \textit{streamcluster} and \textit{gaussian}, \chaining+\prioname\ performs better than \bypassname+\prioname.

\section{Comparison with related work} \label{comparison}
The objectives of \bypassname\ and \prioname\ are similar to the Mostly Clean DRAMCache (MCC) \cite{mostly-clean} and Staged Memory Scheduling (SMS) \cite{sms} respectively. A more comprehensive description of the working and mechanisms of SMS and MCC can be found in Section \ref{chap:related-work}. Here we quantitatively compare our work with MCC (adapted for IHS architecture) and SMS in Figure \ref{results-comparison} which presents performance of (a)CPU and (b)GPU normalized to a IHS with no DRAMCache. 

\begin{figure}[!htb]
	\centering
	\includegraphics{graphs/results-comparison-cpu}
	\includegraphics{graphs/results-comparison-gpu}
	\caption{Comparison of \cachename\ against MCC \cite{mostly-clean} and SMS \cite{sms} (a)CPU (b)GPU}
	\label{results-comparison}
\end{figure}

\par Overall, \bypassname\ performs 7.8\% better than MCC in terms of CPU IPC. This is because when a page (row) is marked as write-through in MCC, all requests to that page are bypassed oblivious of source of the request. The large number of GPU requests to the cache lines in the write-through (clean) pages quickly exhausts the available MSHR entires which leads to the DRAMCache being blocked for subsequent requests. For workloads Qg9, Qg10 and Qg12 the GPU workloads are relatively less intensive and MCC tends to perform slightly better than \bypassname. The GPU performance is comparable for both approaches.
\par The striped bars in Figure \ref{results-comparison} show the performance of the \prioname\ and SMS for (a)CPU and (b)GPU. For SMS implementation, we proportionally scale down the total hardware requirements to the same size as that of \prioname\ (128 entries). We observe that \prioname\ performs on-par with SMS for almost all CPU workloads. For workloads like Qg2 and Qg12 SMS performs marginally better than \prioname\ as it is able to better manage CPU inter-application interference using SJF scheduling. However for the GPU, SMS performs 4\% better than \prioname\ due to its batching algorithm that is able to admit row buffer hit requests into the queues which leads to improved row buffer hit rate (9\% better). Our combined mechanism of \prioname+\bypassname\ proposed in this work performs better then each of prior compared schemes for both CPU and GPU.

\section{Summary}
In this chapter, we evaluated the performance of \cachename\ and its constituent mechanisms and some of their combinations, first on individual CPU and GPU cores and subsequently on IHS architecture as a whole. Finally, we also perform sensitivity studies for capacity, bandwidth and latencies.