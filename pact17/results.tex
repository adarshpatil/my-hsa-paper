\section{Results} \label{results}
In this section, we examine the results of applying our mechanisms to the DRAMCache over an IHS setup. Figure \ref{results-cpu} (a) and (b) shows the ANTT improvement for CPU and GPU respectively for each scheme over the baseline IHS configuration with no DRAMCache. The first bar shows the performance of a naive DRAMCache as discussed in Section \ref{design}. 
\begin{figure*}[!htb]
    \centering
    \includegraphics[scale=1.5]{graphs/results-cpu-paper}
    \label{results-graph}
\end{figure*}
\subsection{PrIS DRAMCache scheduling}
Prioritizing CPU requests at the DRAMCache controller leads to good performance benefits. However, 
\subsection{ByE for Temporal bypass}

\subsection{Chaining for Spatial Occupancy Control}
We empirically determine the low occupancy threshold of the CPU \textit{$l_{cpu}$} in the cache to be 25\% for all our workloads.
