\documentclass[final,t]{beamer}

% poster template
\usepackage[orientation=portrait,size=a0,scale=1.4,debug]{beamerposter}
\usetheme{zurichposter}
\usepackage{graphicx}
\usepackage{varwidth}
\usepackage{multirow}
% references
%\usepackage[bibstyle=authoryear, citestyle=authoryear-comp,%
%hyperref=auto]{biblatex}
%\bibliography{references}

\newcommand*\circled[1]{\tikz[baseline=(char.base)]{
            \node[shape=circle,draw,inner sep=2pt] (char) {#1};}}
%title

% document properties
\title{\LARGE HAShCache: Heterogeneity Aware Shared DRAMCache For Integrated Heterogenous Architectures}
\author{Adarsh Patil, R Govindarajan}
\institute{Department of CSA, Indian Institute of Science, Bangalore }


%------------------------------------------------------------------------------
\begin{document}

% my figures
\input{../figures}

\begin{frame}[t,fragile]{}
\vspace{-1em}
\begin{columns}[t]

% gloablly set small font
\small

%-----------------------------------------------------------------------------
%                                                                     COLUMN 1
% ----------------------------------------------------------------------------
\begin{column}{.47\linewidth}

	% intro HSA
    \begin{exampleblock}{Integrated Heterogenous Systems (IHS) Architecture}
    \vspace{-0.2em}
    \begin{itemize}
    	\item \textit{Throughput-oriented} GPGPU SMs placed alongside \textit{Latency-oriented} CPU cores on-chip
   		\item Shared Physical/Virtual Address Space and a Unified Memory Hierarchy 
		\item Cache Coherent Interconnect
    	\item Improved Programmability
    	\begin{enumerate}[i]
	    	\item Avoids explicit management of data (memcpy, cudamalloc etc.)
	    	\item Allows pointer sharing
	    	\item GPU application data set sizes need not be constrained by memory size
	    	\item Enables HLLs to exploit fine grained parallelism synergistically with CPUs
    	\end{enumerate}
		\item AMD APUs, Intel Iris, NVIDIA Denver
    \end{itemize}
    \end{exampleblock}
% ----------------------------------------------------------------------------    
    % intro D$
    \begin{exampleblock}{Vertically Stacked DRAM}
    \vspace{-0.2em}
    \begin{itemize}
        	\item DRAM Layers stacked using 2.5D interposer or 3D TSV
       		\item Large capacity \textasciitilde order of GBs \\
       				\qquad - can help contain the working sets of IHS workloads
    		\item Very High Bandwidth \textasciitilde 500 GB/s (90GB/s DDR4)\\
    				\qquad - can assist throughput-oriented GPU
    		\item 20 \% reduction in access latency \\
    				\qquad - can enhance CPU performance
        	\item Samsung Wide IO, AMD/Hynix HBM, Intel/Micron HMC
        \end{itemize}
    \end{exampleblock}
% ----------------------------------------------------------------------------        
    % motivation
    \begin{exampleblock}{Motivation}
    \vspace{-0.2em}
    \centering 
    \underline{Performance}

    \begin{itemize}
       	\item When running alone CPU and GPU both gain from addition of DRAM\$
      	\item Effect of Co-running \\
      		\qquad - CPU performance severely degraded, drops 320\% from onlyCPU\\
      		\qquad - GPU remain relatively unaffected, drops 7\% from onlyGPU
      	\item Naive addition of DRAM\$ over IHS \\
      		\qquad - CPU recovers only 42\% of the lost performance over onlyCPU \\
      		\qquad - GPU still performs  15\% better than onlyGPU \\
    \end{itemize}
    \begin{figure}
       \includegraphics[scale=2.1]{../graphs/motivation-cpu}
       \includegraphics[scale=2.1]{../graphs/motivation-gpu}
       \caption{Performance comparison of CPU \& GPU - alone, co-running, with and without DRAM\$}
       \label{fig:motivation}
    \end{figure}
    
    \underline{Causes for sub-optimality of DRAM\$}
    \begin{itemize}
   		\item Increased DRAM\$ access times for CPU despite comparable hit rates
        \item Set contention does not allow GPU to occupy enough cache to use the bandwidth benefit
    \end{itemize}
    
    \begin{figure}
       \includegraphics[scale=2.05]{../graphs/motivation-cpu-cache}
       \includegraphics[scale=2.05]{../graphs/motivation-gpu-cache}
       \caption{(a)CPU D\$ Access Latency and Hit Rates (b)GPU Hit rate with 2-way assoc cache}
       \label{fig:motivation-cpu-cache}
    \end{figure}  

    \end{exampleblock}
     

%-----------------------------------------------------------------------------

    % Design
    \begin{exampleblock}{HAShCache Design for IHS Processors}
    \vspace{-0.2em}
        \quad Large Metadata overhead \\
        	\quad \qquad- Tags-in-DRAM, 128 Byte TAD Units \\
        \quad Set Associativity \\
        	\quad \qquad- Direct Mapped to avoid multiple tag lookups \\
        \quad Miss Penalty \\
        	\quad \qquad- Initiate early access to memory for CPU using a miss predictor \\
        \quad RBH vs BLP addressing scheme \\
	        \quad \qquad- Despite higher BLP, DRAM\$ performs better using RBH addressing\\
        \quad Effect of Interference \\
        	\quad \qquad- DRAM\$ must be designed to try to regain CPU performance lost due to interference\\
    \end{exampleblock}
	
%-----------------------------------------------------------------------------
\iffalse
    % experimental setup
	\begin{exampleblock} {Experimental Setup}
	{
	\footnotesize
	\begin{itemize}
		\item Cycle accurate, modified gem5-gpu simulator
		\item 5 OoO x86-64 cores @2.5GHz running multi-programmed SPEC 2006 workload
		\item Private 32KB I/D L1, 1MB shared split L2 - all coherent
		\item 8 Fermi-like SMs @700MHz running mem-copy elided Rodinia workload
		\item 64KB private L1 non-coherent, 512KB Coherent L2 cache
		\item Stacked DRAM (HMC\_2500\_x64) - 64MB, 2 vault, 8 layers/vault, peak B/W 40GB/s
		\item off-chip DRAM (DDR3\_1600\_x64) - 8GB, 2 channel, 1 rank/channel, peak B/W 25GB/s
	\end{itemize}
	}
	\end{exampleblock}	
\fi	
%-----------------------------------------------------------------------------
	
\begin{exampleblock}{Results}
                \begin{figure}
			    	\includegraphics[scale=2.7]{../graphs/results-cpu}
			    \end{figure}
			    \begin{figure}
			    	\includegraphics[scale=2.7]{../graphs/results-gpu}
			    \end{figure}        			

\end{exampleblock}

\end{column}
%-----------------------------------------------------------------------------
%                                                                     COLUMN 2
% ----------------------------------------------------------------------------
\begin{column}{.47\linewidth}
% ----------------------------------------------------------------------------

	% hsa arch
	\begin{figure}
		\centering
    	\includegraphics[scale=3]{hsa-arch} \\
    	\caption{Architecture of an Integrated Heterogeneous System with a stacked DRAM}
    \end{figure}
% ----------------------------------------------------------------------------   

    % Priority
    \begin{exampleblock}{ \circled{1} Hetereogenity Aware DRAM\$ Scheduling: PrIS}
    \begin{itemize}
    	\item \emph{OBJECTIVE:} Reduce large access latencies for CPU requests at DRAM\$
    	\item Large number of GPU requests $\implies$ queues fill up rapidly $\implies$ CPU request rejected
    	\item GPU requests have good row buffer locality $\implies$ preferentially scheduled $\implies$ large queuing latency for CPU requests
    	\item \emph{Achieved} using \\
    	\qquad - Queue entry reservation for CPU requests when queues reach critical levels \\
    	\qquad - CPU Prioritized FR-FCFS, IHS-aware scheduling algorithm
    \end{itemize} 
	\end{exampleblock}
	    
% ----------------------------------------------------------------------------    
    % chaining
    \begin{exampleblock}{\circled{2} Spatial Occupany Control : \textit{Chaining}}
    \begin{columns} [T]
    \begin{column}{0.6\linewidth}    
    	\begin{itemize}
	    	\item \emph{OBJECTIVE:} Allow GPU to better use DRAM\$ bandwidth
		    \item \emph{Achieved} by providing pseudo-associativity for GPU, thus improving GPU hit rate
		    \item Provides guaranteed minimum occupancy for CPU lines in the cache
	    	\item GPU set conflicts resolved by evicting an adjoining "\textit{chained}" set belonging to the CPU in same row
		    \item Chaining done until minimum CPU occupancy threshold is reached
		    \item Overhead: NIL, uses unused bits in DRAM\$ rows
	    \end{itemize}
	\end{column}
	\begin{column}{0.4\linewidth}
		\scriptsize
		- CPU fill requests always inserted in original set\\
		- GPU fill request inserted as per table below\\
		\begin{table}[]
		\centering
			\begin{tabular}{|c|c|c|c|c|}
\hline
\multirow{2}{*}{\begin{tabular}[c]{@{}l@{}}Threshold \\ Reached?\end{tabular}} & \multirow{2}{*}{\begin{tabular}[c]{@{}l@{}}Original\\ Set\end{tabular}} & \multirow{2}{*}{\begin{tabular}[c]{@{}l@{}}Not Chained\end{tabular}} & \multicolumn{2}{c|}{Chained Set}                                                                                         \\ \cline{4-5} 
                                                                               &                                                                         &                                                                         & CPU                                                        & GPU                                                        \\ \hline
\multirow{2}{*}{No}                                                             & CPU                                                                     & \begin{tabular}[c]{@{}l@{}}replace original\end{tabular}              & \begin{tabular}[c]{@{}l@{}}replace chained\end{tabular}  & \begin{tabular}[c]{@{}l@{}}replace original\end{tabular} \\ \cline{2-5} 
                                                                               & GPU                                                                     & \begin{tabular}[c]{@{}l@{}}Chain to nearest CPU set\end{tabular}   & \begin{tabular}[c]{@{}l@{}}replace chained\end{tabular}  & \begin{tabular}[c]{@{}l@{}}replace original\end{tabular} \\ \hline
\multirow{2}{*}{Yes}                                                             & CPU                                                                     & \begin{tabular}[c]{@{}l@{}}Chain to nearest GPU Set\end{tabular}    & \begin{tabular}[c]{@{}l@{}}do not  insert\end{tabular}    & \begin{tabular}[c]{@{}l@{}}replace chained\end{tabular}  \\ \cline{2-5} 
                                                                               & GPU                                                                     & \begin{tabular}[c]{@{}l@{}}replace original\end{tabular}              & \begin{tabular}[c]{@{}l@{}}replace original\end{tabular} & \begin{tabular}[c]{@{}l@{}}replace original\end{tabular} \\ \hline
\end{tabular}
		\end{table}
	\end{column}
	\end{columns}
	\vspace{\baselineskip}
    \centering
    \begin{figure}
    	\includegraphics[scale=2.2]{chaining}
    	\caption{HAShCache Row Organization and Access Path of a request}
    \end{figure}
    \end{exampleblock}
    
% ----------------------------------------------------------------------------        
    % bypass
    \begin{exampleblock}{\circled{3} Temporal Selective Bypass Enabler : \textit{ByE}}
    	\begin{columns} [T]
    		\begin{column}{0.6\linewidth}
    	    \begin{itemize}
    	    	\item \emph{OBJECTIVE:} Utilize the idle DRAM bandwidth
    	    	\item Bypass CPU requests to clean cache lines and cache misses
    	    	\item \emph{Achieved} using a \textit{Counting Bloom Filter} that tracks dirty lines in cache
    	    	\item Overhead: 256KB (0.4\% of cache capacity)
			    \item Adjoining Fig shows working of HAShCache + ByE
    	    \end{itemize}     			
    		\end{column}
    		\begin{column}{0.4\linewidth}
    			\begin{figure}
					\centering
    			    \includegraphics[scale=1.9]{bloom}
    			\end{figure}
    	    \end{column}
    	\end{columns}
	\end{exampleblock}       
% ----------------------------------------------------------------------------    
\begin{exampleblock}{Conclusion}
\vspace{-0.1em}
\begin{itemize}
\item HAShCache is heterogeneity aware and improves system throughput and achieves better resource utilization
\item Spatial Occupancy Control improves performance of CPU by 44\% by trading off just 6\% of GPU performance
\item Temporal Bypass improves performance of CPU by 48\% while sacrificing just 3\% of GPU performance 
\item Overall, addition of a stacked DRAM\$ improves CPU and GPU performance by 211\% and 20\% over a IHS baseline system with no stacked DRAM
\end{itemize}

\end{exampleblock}
\end{column}

\end{columns}
\iffalse
\begin{columns}[t]
\begin{column}{.97\linewidth}
\begin{exampleblock}{Results \& Conclusion}
	\begin{columns}[t]
		\begin{column}{.32\linewidth}
		    \begin{figure}
		    	\includegraphics[scale=3]{../graphs/results-cpu}
		    \end{figure}
		\end{column}
		\begin{column}{.32\linewidth}
		    \begin{figure}
		    	\includegraphics[scale=3]{../graphs/results-gpu}
		    \end{figure}
		\end{column}		
		\begin{column}{.32\linewidth}
			asfjasdkjf
		\end{column}				
	\end{columns}

\end{exampleblock}
\end{column}
\end{columns}
\fi
\end{frame}

\end{document}
