\section{Motivation} \label{motivation}
As seen in section \ref{introduction}i, incorporating GPU into the same die as CMPs provides for several advantages and the use of this as a general purpose computing resource is gaining traction in applications. However the directions for improving the performance of these integrated contrasting processors is not straight forward. In this work we propose the addition of a large capacity stacked DRAM between CPU and GPU. This capacity will be used as a last level cache before accessing the off-chip DRAM. Adding smaller capacity SRAM caches as shared levels does not contribute much to performance (why?) and DRAMs are better suited to handle larger number of requests and interference (why?). And hence the DRAMCache is the first level shared memory capacity in the hierarchy of the 2 processors.
Due to the requirement of large storage, of order of the last level SRAM caches, prior works have placed metadata alongside the data in the DRAM devices and focused on improving access latencies for the accesses.
Set associativity usually improves cache hit rates by reducing conflict misses. However in DRAMCaches associativity comes at a cost of increased hit latencies. Since each way comparison would require a tag to be burst out of the DRAM. Metadata is placed in DRAM
\makeatletter
\newcounter{groupcount}
\pgfplotsset{
    draw group line/.style n args={5}{
        after end axis/.append code={
            \setcounter{groupcount}{0}
            \pgfplotstableforeachcolumnelement{#1}\of\datatable\as\cell{%
                \def\temp{#2}
                \ifx\temp\cell
                    \ifnum\thegroupcount=0
                        \stepcounter{groupcount}
                        \pgfplotstablegetelem{\pgfplotstablerow}{[index]0}\of\datatable
                        \coordinate [yshift=#4] (startgroup) at (axis cs:\pgfplotsretval,0);
                    \else
                        \pgfplotstablegetelem{\pgfplotstablerow}{[index]0}\of\datatable
                        \coordinate [yshift=#4] (endgroup) at (axis cs:\pgfplotsretval,0);
                    \fi
                \else
                    \ifnum\thegroupcount=1
                        \setcounter{groupcount}{0}
                        \draw [
                            shorten >=-#5,
                            shorten <=-#5
                        ] (startgroup) -- node [anchor=north] {#3} (endgroup);
                    \fi
                \fi
            }
            \ifnum\thegroupcount=1
                        \setcounter{groupcount}{0}
                        \draw [
                            shorten >=-#5,
                            shorten <=-#5
                        ] (startgroup) -- node [anchor=north] {#3} (endgroup);
            \fi
        }
    }
}
\makeatother

\pgfplotstableread{
1   19.178  26.027  8.219   6.849   39.726  1
2   54.795  21.918  4.110   6.849   12.329  1
3   28.767  16.438  6.849   8.219   39.726  1
4   63.014  2.740   2.740   2.740   28.767  2
5   90.411  1.370   6.849   0.000   1.370  2
6   15.068  2.740   16.438  8.219   57.534  2
7   67.123  0.000   0.000   1.000   32.877  3
8   72.603  6.849   5.479   5.000   15.068  3
9   56.164  12.329  6.849   4.110   20.548  3
10  50.685  4.110   8.219   1.370   35.616  3
}\datatable

\begin{tikzpicture}
\begin{axis}[
    ylabel=label,
    xtick=data,
    xticklabels={S1,S2,S3,S4,S5,S6,S7,S8,S9,S10},
    enlarge y limits=false,
    enlarge x limits=0.1,
    ymin=0,ymax=100,
    ybar stacked,
    bar width=10pt,
    legend style={
      font=\footnotesize,
      cells={anchor=west},
      legend columns=5,
      at={(0.3,-0.20)},
      anchor=north,
      /tikz/every even column/.append style={column sep=0.2cm}
    },
]
\addplot table[x index=0,y index=1] \datatable;
\addplot table[x index=0,y index=2] \datatable;
\addplot table[x index=0,y index=3] \datatable;
\legend{Far,Near,Here}
\end{axis}
\begin{axis}[
    ylabel=label,
    xtick=data,
    xticklabels={S1,S2,S3,S4,S5,S6,S7,S8,S9,S10},
    enlarge y limits=false,
    enlarge x limits=0.1,
    ymin=0,ymax=100,
    legend style={
      font=\footnotesize,
      cells={anchor=west},
      legend columns=5,
      at={(0.71,-0.20)},
      anchor=north,
      /tikz/every even column/.append style={column sep=0.2cm}
    },
    draw group line={[index]6}{1}{X Group}{-3.5ex}{7pt},
    draw group line={[index]6}{2}{Y Group}{-3.5ex}{7pt},
    draw group line={[index]6}{3}{Z Group}{-3.5ex}{7pt}
]

\addplot+[forget plot] table[x index=0,y index=4, restrict x to domain=0:3] \datatable;
\addplot+[forget plot] table[x index=0,y index=4, restrict x to domain=4:6] \datatable;
\addplot+ table[x index=0,y index=4, restrict x to domain=7:10] \datatable;
\legend{There}
\end{axis}
\end{tikzpicture}