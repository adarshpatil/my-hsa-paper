\section{Motivation} \label{motivation}
As seen in section \ref{introduction}, the large capacity provided by the stacked DRAM are in line with the working set requirements of IHS processors. Using this capacity as a hardware managed cache could provide performance gains without any application modifications. The bandwidth benefits and modestly improved latency provided by the DRAMCache help improve performance of IHS processors over on-chip SRAM caches of reasonable sizes \cite{amd-exascale1}. 

In this paper, we assume a cache organization similar to Alloy cache \cite{alloy} with a block size of 128 bytes and study the problems and challenges in designing the DRAM cache for IHS architecture. We justify this design decisions in the following section. 
Each CPU core has a private L1 cache and a shared split L2 cache across the CPU cores. The GPU cores have private L1 and shared L2 cache among themselves.  
The stacked DRAMCache (of size 64MB) is the first level shared cache across CPU cores and GPU cores. In our experiments, we consider a multi-programmed workload on the CPU cores and a GPGPU application on a single CPU core and multiple SMs. It should be noted here that the GPU application has a CPU component and alternates execution on CPU core and the SMs (kernel execution), and our study on the interference due to co-running CPU and GPU application, reports performance numbers of the GPU application when the GPU kernel is running on the SMs.  
Additional details relating the experimental methodology and workloads are described in Section \ref{methodology}

First we evaluate the benefits of having a large DRAM cache in IHS architecture. Figure \ref{motivation} presents the performance
improvement (in terms of Average Normalized Turn-around Time (ANTT) \cite{antt} for the CPU applications with and without
the stacked DRAM cache, and when they are running with and without the GPU application. As can be seen from the figure,
the performance of CPU applications experience significant improvement (by almost a factor of x.y) with the introduction 
of the DRAM cache, but when they are run without the GPU application. Co-running with GPU application, however results
only in XX\% improvement in ANTT. This loss in performance is essentially due to co-running GPU application, which hampers 
the performance of latency-sensitive CPU application heavily. Figures~\ref{???} and \ref{???} present the cache hit rate
and the average hit time in DRAM Cache (or you can present average memory access latency).  We can see that the presence
of GPU applications significantly increase average hit time (by a factor of x.y), while hit rates are impacted only marginally 
(only by XX\%). The reasons for the increase in average hit time and the decrease in ANTT is primarily due to the large
number of GPU memory requests flooding the DRAMCache controller when the GPU application is co-run with the CPU applications.

Next, let us study the impact of co-running on the GPU applications.  Figure~\ref{???}(b) again presents GPU ANTT improvement\footnote{In this 
paper we report CPU ANTT and GPU ANTT (and the improvement in them) independently. This is done to identify the impact of CPU 
and GPU applications on each other, as well as to understand the impact of the proposed modifications on each of the application
type.} for various workloads. 
We observe the following: 
(i)  the introduction of DRAMCache improves the performance of GPU application by xx\% and yy\% with and without co-running CPU applications;
and (ii) Co-running CPU applications has only a minor impact on the GPU ANTT (with or without the DRAM Cache). 
%% one sentence on why in some of the workloads there is no improvement in ANTT. 
%% If required you can present the GPU hit rate and average hit time in DRAM Cache for GPU appln. 

%% Do you want to present the data to establish that the GPU application only occupies certain part of the DRAM cache 
%% and is unable to use beyond this due to cache contention, making a case of pseudo associativity?
%% you shd. do this without getting into chaining, etc. 

Based on the above motivation study we conclude that there are significant performance benefits that could be obtain 
with the introduction of the stacked DRAMCache for the latency-sensitive CPU applications. However, in a naive implementation
of the DRAM cache, these benefits can be undone by the co-running GPU application.  Hence it is important to carefully 
architect the DRAMCache design so that CPU applications are not hampered due to interference of co-running GPU application.
This requires the design to be aware of the heterogeneity of the applications (CPU vs GPU) and their demands on the 
memory hierarchy.  Hard partitioning of DRAMCache may not be an effective design as it leads to under-utilization of the large capacity 
stacked DRAM, as the interference due to GPU application primarily happens when the kernel is running on the SMs. 
Further, effectively utilizing the under-utilized main memory bandwidth ~\cite{Nagendra's Memsys paper, Mainak's HPCA paper}
is also important to achieve higher performance.  Further the design should ensure that the CPU applications and the 
GPU application are able to utilize the large capacity of the stacked DRAM effectively to meet the working set 
requirements of the respective applications in the best possible way.

\begin{figure}[htbp]
   \includegraphics[scale=1.0]{graphs/motivation-cpu}
   \includegraphics[scale=1.0]{graphs/motivation-gpu}
   \caption{ANTT Improvement of CPU and GPU}
   \label{fig:motivation}
\end{figure}
