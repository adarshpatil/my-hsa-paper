\section{Experimental Setup \& Methodology} \label{methodology}
\par \textbf{Simulator:} We evaluate the performance of the DRAMCache over a IHS Processor using a cycle accurate simulator gem5-gpu \cite{gem5-gpu} which is configured to simulate cache coherent unified CPUs and GPUs. The GPU cache hierarchy has per SM private L1  that are non-inclusive of the shared GPU L2 cache and can hold stale data. However, GPU L2 cache is coherent with all levels of the CPU hierarchy. The CPU and GPU cache are kept coherent using the VI\_Hammer protocol. Table \ref{configuration} provides the setup details.\\
\par \textbf{Workloads:} Applications having high and medium memory intensive behaviours from SPEC CPU2006 suite \cite{spec2006} were chosen to form a multi-programmed workload. We use the Rodinia benchmark suite \cite{rodinia} to represent applications that use the GPU offload for execution. These Rodinia applications are modified to elide the \textit{memcpy} api calls so as to run with unified virtual and physical address spaces. Combinations of quad-core multi-programmed workloads were coupled with a full Rodinia benchmark to create a representative mix of applications that would run in an IHS system. These composite workloads embody different levels of total memory intensity produced by the CPU and GPU cores. We also measure the footprint of these workloads in terms of number of unique 128B cache blocks accessed at the DRAMCache. The memory footprints from 70MB to 650MB for quad-core CPU workloads and from 5.5MB to 135MB for the GPU application. The smaller footprints forces us to use a smaller stacked DRAMCache capacity but the observations and techniques proposed can be generalized to larger stacked DRAMCaches.\\
\par \textbf{DRAMCache Design:} The DRAMCache we evaluate here is the first level shared cache between the 2 split cache hierarchies of CPUs and GPUs. Access to memory is interleaved across multiple memory controllers. A biased interleaving due to uniformly strided address patterns is restricted by using a basic XOR-based address hashing mechanism. Each memory controller manages a 4GB DDR3 memory device and a 32MB stacked DRAM vault. The stacked DRAM vault caches data from the corresponding 4GB memory device that the controller is responsible for. This setup ensures that there are no cross bus requests between controllers. Since our setup has the limitations of having a 1-to-1 channel mapping between a stacked DRAM vault and a off-chip DRAM channel, our model does not provide sufficient channel level parallelism as is quintessential in a stacked DRAM device, where the large number of vaults and TSVs allow the stacked DRAM to provide a larger bandwidth. To circumvent this limitation we increase the number of layers (ranks) to provide higher amount of parallelism in our stacked DRAM. However, the the bank capacity is retained as would be in a standard stacked DRAM. We faithfully model a Fill Queue \cite{dca} for fill requests to insert data into the cache on the return path from main memory. We also model MSHR and WriteBuffers with their associated latencies to realize the precise working of caches. Our DRAMCache also respects all significant timing and functional parameters using the DRAMCtrl \cite{dramctrl} model. The DRAMCache is non-inclusive \cite{coherence-dramcache} of the  L2 caches above it and uses a write-noallocate policy.\\
\par \textbf{Simulation Methodology:} We fast-forward the initialization phase of each workloads up until just before the launch of the first kernel of the GPU program. We ensure that each core is fast-forwarded by atleast 2 Billion instructions and in total each quad-core workload by 20 billion instructions on average in the fast-forward phase. This is accomplished by adding no-ops to the Rodinia benchmarks for the duration until the initialization quota of the SPEC programs is complete. We then warm the cache until the fastest core completes 250 million instructions. During the warmup phase the GPU program is not executed. Timing simulations were then run for 250 million instructions for each CPU core for the quad-core workload workloads. As is the norm, when a core finishes its quota of instructions, it continues to execute until all the cores have completed. \\
The Rodinia application uses an extra CPU core and offloads to the integrated GPU. These applications are modified to execute in a \textit{conditioned loop} such that there is no corruption of data structures in the program. The \textit{conditioned loop} is run infinitely and represents the region of interest of the Rodinia benchmark. This region includes sections of CPU activity and GPU offloads in the execution, as is typical of a HSA program which will exhibit an interleave of offloaded regions and serial CPU sections. However, only the performance statistics for the first execution of the \textit{conditioned loop} in the program are considered. In cases where the ROI is longer than the complete run of the CPU workloads, the statistics for the last completed kernel are used. 
\par \textbf{Performance Metric} We report system performance by adapting the ANTT \cite{antt} metric for Integrated Heterogeneous Systems and define it as follows
\begin{equation}
ANTT_{IHS_{CPU}} = \frac{1}{n_{cpu}} \sum_{i=1}^{n_{cpu}} \frac{CPI_i^{CPU_{IHS}}}{CPI_i^{SP}},
ANTT_{IHS_{GPU}} = \frac{CPI^{GPU_{IHS}}}{CPI^{GPU}}
\end{equation}
where ${CPI_i^{CPU_{IHS}}}$ and $CPI^{GPU_{IHS}}$ denote the cycles per instruction achieved by the $i^{th}$ CPU and the GPU when running in a IHS setup respectively; whereas $CPI_i^{SP}$ and $CPI^{GPU}$ denote the same when $i^{th}$ CPU and the GPU are running standalone respectively. The overall system performance is then given as $ANTT_{IHS} = ANTT_{IHS_{CPU}} + ANTT_{IHS_{GPU}} $

\begin{table}[h!]
  \centering
  \begin{tabular}{{@{}ll@{}}}
    \toprule
    CPU Core 	& five 4-wide OoO x86 cores @2.5 GHz \\
    \midrule
    CPU Caches 	& 32KB 8-way split I/D private L1 Cache, \\ 
    		   	& 1M 8-way shared split L2 Cache, 128B lines \\
    \midrule
    GPU Core 	& eight Fermi SMs@700MHz, 2x2 GTO warp sched\\
    			& upto 32 threadblocks, 64 warps of 32 threads, \\
    			& 64K registers, 96KB scratch memory \\
    \midrule
    GPU Caches 	& 64KB 4-way private L1 cache,\\ 
               	& 512KB, 16-way assoc L2 Coherent Cache \\
    \midrule
    Stacked     & 2 vaults, HMC\_2500\_x64, 2KB Page \\
	DRAM		& $t_{CL}$-$t_{RCD}$-$t_{RP}$-$t_{RAS}$=9.9ns-10.2ns-7.7ns-21.6ns\\
    			& 8 layers/vault, 4 banks/layer\\
    			& 64 byte burst size, Peak bandwidth 19.5GB/s \\
				& Refresh related: $t_{REFI}$=3.9us $t_{RFC}$=59us \\
    \midrule
    Off-chip 	& 2 channels, DDR3\_1600\_x64, 1KB page \\
    DRAM		& $t_{CL}$-$t_{RCD}$-$t_{RP}$-$t_{RAS}$=13.75ns-13.75ns-13.75ns-35ns\\
    			& 1 rank/channel, 8 banks/rank\\
    			& 64 byte burst size, Peak bandwidth 12.5GB/s \\
    			& Refresh related: $t_{REFI}$=7.8us $t_{RFC}$=260us \\
    			  
    \bottomrule
  \end{tabular}
  \caption{Configuration of the simulated system}
  \label{configuration}
\end{table}

\begin{table}[h!]
  \centering
  \begin{tabular}{{|l|l|l|}}
    \hline
    \textbf{Name} & \textbf{Multi-program SPEC2006} & \textbf{Rodinia}\\
    \hline
    Qg1& cactusADM;gcc;bzip2;sphinx3 & needle\\
    \hline
    Qg2 & astar;mcf;gcc;bzip2 & needle\\
    \hline
    Qg3 & gcc;libquantum;leslie3d;bwaves & needle\\
    \hline
    Qg4 & astar;soplex;cactusADM;libquantum & k-means\\
    \hline
    Qg5 & milc;mcf;libquantum;bzip2 & k-means\\
    \hline
    Qg6 & bzip2;gobmk;hmmer;sphinx3 & k-means\\
    \hline
    Qg7 & soplex;milc;cactusADM;libquantum & gaussian\\
    \hline
    Qg8 & milc;libquantum;gobmk;leslie3d & gaussian\\
    \hline
    Qg9 & astar;milc;gcc;leslie3d & hotspot\\
    \hline
    Qg10 & gcc;gobmk;leslie3d;sphinx3 & hotspot\\
    \hline
    Qg11 & astar;cactusADM;libquantum;sphinx3 & srad\\
    \hline
    Qg12 & astar;mcf;gobmk;sphinx3 & scluster\\
    \hline
    Qg13 & astar;cactusADM;libquantum;sphinx3 & lud\\
    \hline
  \end{tabular}
  \caption{Workloads}
  \label{workloads}
\end{table}
