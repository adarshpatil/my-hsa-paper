\section{Introduction}\label{introduction}

% HSA Arch diagram
\usetikzlibrary{shapes,arrows}
\usetikzlibrary{backgrounds}
\usetikzlibrary{matrix, positioning, fit}
\usetikzlibrary{patterns}
\definecolor{forestgreen}{rgb}{0.0, 0.5, 0.0}

\tikzstyle{smallcircle} =[fill=black!100, text=white, circle, inner sep=1pt, minimum size=0.1em]
\tikzstyle{ourcircle} = [draw, semicircle, inner sep=0pt,minimum size=2pt]
\tikzstyle{dullblock} = [draw, fill=black!20, circle, minimum height=2em, minimum width=2em]
\tikzstyle{block} = [draw=black, rounded corners, thick, line width=0.3mm, rectangle, minimum height=10em, minimum width=7em]
\tikzstyle{blocksmall} = [draw=black, thick, line width=0.5mm, rectangle, minimum height=6em, minimum width=3em, fill=white]
\tikzstyle{group} = [draw=black, line width=0.3mm, rectangle, minimum height=2em, minimum width=2em]
\tikzstyle{textblock} = [draw, fill=blue!20, rectangle, rounded corners]
\tikzstyle{plain} = []
\tikzstyle{input} = [draw, thick, fill=blue!20, circle, minimum size=1pt]
\tikzstyle{output} = [draw, fill=blue!20, circle, minimum size=1pt]
\tikzstyle{pinstyle} = [pin edge={to-,thin,black}]
\tikzset{toprule/.style={%
        execute at end cell={%
            \draw [line cap=rect,#1] (\tikzmatrixname-\the\pgfmatrixcurrentrow-\the\pgfmatrixcurrentcolumn.north west) -- (\tikzmatrixname-\the\pgfmatrixcurrentrow-\the\pgfmatrixcurrentcolumn.north east);%
        }
    },
    bottomrule/.style={%
        execute at end cell={%
            \draw [line cap=rect,#1] (\tikzmatrixname-\the\pgfmatrixcurrentrow-\the\pgfmatrixcurrentcolumn.south west) -- (\tikzmatrixname-\the\pgfmatrixcurrentrow-\the\pgfmatrixcurrentcolumn.south east);%
        }
    }
}

\newcommand{\hsacpu}[0]{
    \begin{tikzpicture}[auto, >=latex']
        %CPU
        \node [blocksmall, minimum height=1em, minimum width=1em] (cpu1) {\begin{varwidth}{4cm}\centering{CPU\\ Core 1} \end{varwidth}};
        \node [blocksmall, above = 0.1cm of cpu1, minimum height=1em, minimum width=1em] (l11){\begin{varwidth}{4cm} \centering{L1} \end{varwidth}};
        
        %\node [blocksmall, right = 0.1cm of cpu1, minimum height=1em, minimum width=1em] (cpu2) {\begin{varwidth}{4cm} \centering{CPU\\ Core 2} \end{varwidth}};
        %\node [blocksmall, above = 0.1cm of cpu2, minimum height=1em, minimum width=1em] (l12){\begin{varwidth}{4cm} \centering{L1} \end{varwidth}};
        
        \node [smallcircle, right=0.15cm of cpu1] (dot1) {};
        \node [smallcircle, right=0.05cm of dot1] (dot2) {};
        \node [smallcircle, right=0.05cm of dot2] (dot3) {};
        
        %\node [blocksmall, right = 0.1cm of cpu2, minimum height=1em] (cpu3){\begin{varwidth}{4cm} \centering{CPU\\ Core 3} \end{varwidth}};
        %\node [blocksmall, above = 0.1cm of cpu3, minimum height=1em] (l13){\begin{varwidth}{4cm} \centering{L1} \end{varwidth}};
        
        \node [blocksmall, right = 0.15cm of dot3, minimum height=1em, minimum width=1em] (cpu4){\begin{varwidth}{4cm} \centering{CPU\\ Core N} \end{varwidth}};
        \node [blocksmall, above = 0.1cm of cpu4, minimum height=1em, minimum width=1em] (l14){\begin{varwidth}{4cm} \centering{L1} \end{varwidth}};
        \node [blocksmall, above = 2.01cm of dot3, minimum height=3em, minimum width=6em] (l24){\begin{varwidth}{4cm} \centering{L2} \end{varwidth}};
        
       % \begin{scope}[on background layer]
            %\node[group, inner sep = 3pt, line width=0.3mm, draw=red, yshift=1em, fit=(cpu1)(cpu2)(cpu3)(cpu4)(l11)(l12)(l13)(l14)(l24)] (grpcpu) {};
       % \end{scope}
        
        
        %GPU
        \node [blocksmall, right = 0.6cm of cpu4, minimum height=1.5em, minimum width=1em] (sm1) {CU};
        \node [blocksmall, above = 0.1cm of sm1, minimum height=1pt, minimum width=1em] (sm1l1) {\scriptsize{L1}};
        \node [blocksmall, right = 0.15cm of sm1, minimum height=1.5em, minimum width=1em] (sm2) {CU};
        \node [blocksmall, above = 0.1cm of sm2, minimum height=1em, minimum width=1em] (sm2l1) {\scriptsize{L1}};
        \node [blocksmall, right = 0.15cm of sm2, minimum height=1.5em, minimum width=1em] (sm3) {CU};
        \node [blocksmall, above = 0.1cm of sm3, minimum height=1em, minimum width=1em] (sm3l1) {\scriptsize{L1}};
        \node [blocksmall, right = 0.15cm of sm3, minimum height=1.5em, minimum width=1em] (sm4) {CU};
        \node [blocksmall, above = 0.1cm of sm4, minimum height=1em, minimum width=1em] (sm4l1) {\scriptsize{L1}};
        \node [blocksmall, above = 2.3cm of sm1, minimum height=1.5em, minimum width=1em] (sm5) {CU};
        \node [blocksmall, below = 0.1cm of sm5, minimum height=1em, minimum width=1em] (sm5l1) {\scriptsize{L1}};
        \node [blocksmall, right = 0.15cm of sm5, minimum height=1.5em, minimum width=1em] (sm6) {CU};
        \node [blocksmall, below = 0.1cm of sm6, minimum height=1em, minimum width=1em] (sm6l1) {\scriptsize{L1}};
        \node [blocksmall, right = 0.15cm of sm6, minimum height=1.5em, minimum width=1em] (sm7) {CU};
        \node [blocksmall, below = 0.1cm of sm7, minimum height=1em, minimum width=1em] (sm7l1) {\scriptsize{L1}};
        \node [blocksmall, right = 0.15cm of sm7, minimum height=1.5em, minimum width=1em] (sm8) {CU};
        \node [blocksmall, below = 0.1cm of sm8, minimum height=1em, minimum width=1em] (sm8l1) {\scriptsize{L1}};  
        
        \node [blocksmall, minimum height=1.1em, minimum width=6em] at (5.6,1.45)
        (l2gpu) {\begin{varwidth}{4cm} \centering{\small{Coherent GPU \\L2}}\end{varwidth}};
        
         %\begin{scope}[on background layer]
            \node[group, inner sep = 3pt, line width=0.3mm, draw=red, fit=(sm1)(sm1l1)(sm2)(sm2l1)(sm3)(sm3l1)(sm4)(sm4l1)(sm5)(sm5l1)(sm6)(sm6l1)(sm7)(sm7l1)(sm8)(sm8l1)(l2gpu)] (grpgpu) {};
         %\end{scope}                        
        
        
        %Interconnect
        \node [blocksmall, minimum height=1.5em, minimum width=10.7em] at (1.5,1.5)
        (interconnect) {Cache Coherent Interconnect};
                
        %Connections to interconnect
        \draw [-,very thick] (l11.north) to ++(0, 0.15); 
        \draw [-, very thick] (l24.south) to ++(0, -0.3);
        \draw [-, very thick] (l2gpu.west) to ++(-0.82,0);
        \draw [-, very thick] (l14.north) to ++(0, 0.15);
        %\draw [-, very thick] (l13.north) to ++(0, 0.15);
        %\draw [-, very thick] (l12.north) to ++(0, 0.15);
        
        % Memory Controller
        \node [blocksmall, minimum height=1em, minimum width=9em, below=0.1cm of cpu4, xshift=-2em] (mc1){\begin{varwidth}{4cm} \centering{Memory Controller} \end{varwidth}};
        \node [blocksmall, minimum height=1em, minimum width=9em, right=0.05cm of mc1] (mc2){\begin{varwidth}{4cm} \centering{Memory Controller} \end{varwidth}};  
        
        %\draw [-, very thick] (-0.78,-0.6) to (-0.78, 1.5) to (interconnect.west);
        \draw [-, very thick] (2.8,-0.6) to (2.8, 1.25);
        \draw [-, very thick] (3.07,-0.6) to (3.07, 1.25);
        
        %Chip
        \begin{scope}[on background layer]
             \node[group, inner sep = 4pt, line width=0.6mm, fill=cyan!70, fit=(cpu1)(cpu4)(l11)(l14)(l24)(interconnect)(mc1)(mc2)(grpgpu)] (grpchip) {};
        \end{scope}
        
        %DRAM Cache  
        \node [block, draw=red, dashed, minimum height=7em, minimum width=4em] at (0,6.2) (sid){\begin{varwidth}{4cm} \centering{\footnotesize{Silicon\\ interposed\\ DRAM}} \end{varwidth}};
        
        \node [block, draw=red, right = 2.1cm of sid, dashed, minimum height=7em, minimum width=4em] (sid1){\begin{varwidth}{4cm} \centering{\footnotesize{Silicon\\ interposed\\ DRAM}} \end{varwidth}};
        
        \node [block, draw=red, minimum height=0.1pt, rounded corners=1pt, inner sep=1pt, minimum width=14em] at (1.77, 4.8) (blk1){\begin{varwidth}{4cm} \end{varwidth}};
        
        \node [block, draw=black, above = 0.05cm of blk1, minimum height=0.1em, minimum width=1em] (cpud){\begin{varwidth}{4cm} \centering{\scriptsize{CPU-GPU Die}} \end{varwidth}};
        
        \node [block, draw=red, above = 0.2cm of cpud, minimum height=4em, dashed, minimum width=4em] (vsd) {\begin{varwidth}{4cm} \centering{\footnotesize{Vertically\\ Stacked\\ DRAM\\}} \end{varwidth}};
        
        \node [blocksmall, draw=white, above = 0.01cm of sid.north east, minimum height=1em, minimum width=0.2em] (lbl3){\footnotesize{On Package region}};
        
        \node[group, inner sep = 3pt, rounded corners, fit=(sid)(vsd)(sid1)(cpud)(blk1)(lbl3)] (g1) {};
        
        % TSV        
        \draw [-,very thick] (1.35,5.38) to (1.35,5.6);
        \draw [-,very thick] (1.65,5.38) to (1.65,5.6);
        \draw [-,very thick] (1.95,5.38) to (1.95,5.6);
        \draw [-,very thick] (2.25,5.38) to (2.25,5.6);
        
        % semicircles
        \node [smallcircle] at (-0.43,4.9) (s4) {};
        \node [smallcircle] at (-0.13,4.9) (s1) {};
        \node [smallcircle] at (0.17,4.9) (s2) {};
        \node [smallcircle] at (0.47,4.9) (s3) {};
        %\node [smallcircle] at (0.7,4.9) (s3) {};
        %\node [smallcircle] at (1.0,4.9) (s4) {};

        \node [smallcircle] at (1,4.9) (s5) {};
        \node [smallcircle] at (1.3,4.9) (s6) {};
        \node [smallcircle] at (1.6,4.9) (s7) {};
        \node [smallcircle] at (1.9,4.9) (s8) {};
        \node [smallcircle] at (2.2,4.9) (s9) {};
        \node [smallcircle] at (2.5,4.9) (s10) {};

        \node [smallcircle] at (3.1,4.9) (s11) {};
        \node [smallcircle] at (3.4,4.9) (s12) {};
        \node [smallcircle] at (3.7,4.9) (s13) {};
        \node [smallcircle] at (4,4.9) (s14) {};
        %\node [smallcircle] at (4.3,4.9) (s15) {};
        
        % GPU L1 to L2 connections
        \draw [-,very thick] (sm1l1.north east) -- (l2gpu);
        \draw [-,very thick] (sm2l1.north) -- (l2gpu);
        \draw [-,very thick] (sm3l1.north) -- (l2gpu);
        \draw [-,very thick] (sm4l1.north) -- (l2gpu);
        \draw [-,very thick] (sm5l1.south east) -- (l2gpu);
        \draw [-,very thick] (sm6l1.south) -- (l2gpu);
        \draw [-,very thick] (sm7l1.south) -- (l2gpu);                
        \draw [-,very thick] (sm8l1.south) -- (l2gpu);
        
        % Off package Region
        \node [blocksmall, draw=blue, minimum height=0.5em, minimum width=0.5em] at (5.65,5.15)(mm1){};
        \node [blocksmall, draw=blue, above = 0.15cm of mm1, minimum height=0.5em, minimum width=0.5em] (mm2){};
        \node [blocksmall, draw=blue, above = 0.15cm of mm2, minimum height=0.5em, minimum width=0.5em] (mm3){};
        \node [blocksmall, draw=blue, above = 0.15cm of mm3, minimum height=0.5em, minimum width=0.5em] (mm4){};
        \node [blocksmall, draw=blue, above = 0.15cm of mm4, minimum height=0.5em, minimum width=0.5em] (mm5){};
        %\node [blocksmall, draw=blue, above = 0.2cm of mm5, minimum height=0.5em, minimum width=0.5em] (mm6){};

        \node [blocksmall, draw=blue, minimum height=0.5em, right = 0.55cm of mm2, minimum width=0.5em] (mm7){};
        \node [blocksmall, draw=blue, above = 0.15cm of mm7, minimum height=0.5em, minimum width=0.5em] (mm8){};
        \node [blocksmall, draw=blue, above = 0.15cm of mm8, minimum height=0.5em, minimum width=0.5em] (mm9){};
        \node [blocksmall, draw=blue, above = 0.15cm of mm9, minimum height=0.5em, minimum width=0.5em] (mm10){};
        \node [blocksmall, draw=blue, above = 0.15cm of mm10, minimum height=0.5em, minimum width=0.5em] (mm11){};
       % \node [blocksmall, draw=blue, above = 0.2cm of mm11, minimum height=0.5em, minimum width=0.5em] (mm12){};
        \begin{scope}[on background layer]
            \node [blocksmall, draw=white, below = 0.3cm of mm7, minimum height=1em, minimum width=1.2em] (lbl1){\scriptsize{DIMMs}};
        \end{scope}
        \begin{scope}[on background layer]
            \node [blocksmall, draw=white, above = 0.21cm of mm5.north east, minimum height=0.1em, minimum width=0.1em] (lbl2){\begin{varwidth}{4cm} \scriptsize{Off Package\\region}\end{varwidth}};
       \end{scope}
        
        % Memory Grouping
        \begin{scope}[on background layer]
              \node[group, minimum width=0.1em, minimum height=0.1em, fill=black!30, fit=(mm1)(mm2)(mm3)(mm4)(mm5)] (mmgrp1) {};
        \end{scope}
        \begin{scope}[on background layer]
              \node[group, minimum width=0.1em, minimum height=0.1em, fill=black!30, fit=(mm7)(mm8)(mm9)(mm10)(mm11)] (mmgrp2) {};
        \end{scope}
        \node[group, inner sep = 3pt, rounded corners, line width=0.3mm, fit=(mmgrp1)(mmgrp2)(lbl1)(lbl2)] (mmgrp3) {};
        %\node[group, inner sep = 3pt, line width=0.3mm, fit=(mmgrp1)(mmgrp2)] (mmgrp3) {};
        
        %\draw [-,very thick] (7.9,6.5) to (7.9, 6.3) to (11.3,6.3) to (11.3,6.5);
        %\draw [-,very thick] (8.5,5.35) to (8.5, 5.15) to (11.9,5.15) to (11.9,5.35);
        \draw [-,very thick] (5.9, 5.1) to (6.05, 5.1) to (6.05, 7.0) to (5.9, 7.0);
        \draw [-,very thick] (6.73, 5.55) to (6.87, 5.55) to (6.87, 7.4) to (6.73, 7.4);
        
        % Big Arrows
        \node[single arrow, draw, below = 0.1mm of cpud.south, xshift=1em,
        yshift=-1.8em, rotate=-90, minimum height=4.24em, minimum width=3em, line width=0.5mm, fill=white](arrow1){};
        \node[double arrow, draw, right = 0.1mm of sid1.east,
        yshift=0em, rotate=0, minimum height=3em, minimum width=1.3em, line width=0.5mm, fill=white](arrow2){};
        
        % Labels
        \node [blocksmall, draw=white, minimum height=1em, minimum width=1.2em] at (0.4,4.0)
        (lbl4){\begin{varwidth}{4cm} \centering{\scriptsize{Through Silicon\\Vias}}\end{varwidth}};
        \node [blocksmall, draw=white, right=1.2cm of lbl4, minimum height=1em, minimum width=1.2em] (lbl5) {\begin{varwidth}{4cm} \centering{\scriptsize{Silicon\\Interposer}}\end{varwidth}};
        \node [blocksmall, draw=white, right=0.1cm of lbl5, minimum height=1em, minimum width=1.2em, yshift=0.4em, inner sep=2pt] (lbl6) {\begin{varwidth}{4cm} \centering{\scriptsize{External Memory \\Interface (e.g.DDR)}} \end{varwidth}};
        
        % Label Arrows
        \draw[->, very thick] (lbl5.north) to (blk1);
        \draw[<-, very thick] (4.6, 5.8) to [in=90,out=-90] ++(0,-1.3);
        \draw[<-, very thick] (1.32,5.49) to [in=0,out=180] ++(-0.7,0) to [in=90,out=-90] ++(0,-1.12);
        
    \end{tikzpicture}
}

\begin{figure*}[!htb]
    \centering
    \hsacpu
    \caption{Architecture of a Integrated Heterogeneous System with DRAM Stacking}
    \label{hsa-arch}
\end{figure*}


\par The remarkable advances in computing power of the modern microprocessor over the last few decades can predominantly be attributed to Moore's Law and advances in manufacturing technology that has allowed shrinking of transistor sizes. Miniaturization of transistors has allowed addition of specialized on-chip hardware circuitry for acceleration units. A study in 2010 by Koomey et. al \cite{koomey} found that the amount of computation that could be done per unit of energy doubled about every 18 months. However, to reach exascale and beyond requires a thousand fold decrease in energy consumed per flop computed. Graphics processing units (GPUs) have evolved from being fixed-function pipelines and are being used to accelerate data parallel code for general purpose (GP) applications. Compared with multi-core CPUs, GPGPUs offer the potential for better performance at lower energy. Traditionally these discrete processors have had their own independent memory systems. To take advantage of the discrete GPUs, the CPU must copy data to the GPUs memory and back. This data movement is wasteful and expends energy while also adding latency as the transfer happens over a slower PCIe bus. The separate address spaces and complex programming models that need to manage 2 sets of data further impede expansion of the workloads that benefit from GPGPU computing. 
\par Modern processor chips have heterogeneous processors which integrate a lower capacity GPUs on-die allowing for graphics rendering only. In view of the widespread use of the GPU for general purpose applications, processor manufacturers including AMD\cite{amd-apu}, Intel\cite{inteliris}, and NVIDIA\cite{denver} are beginning to allow general purpose OpenCL/CUDA programs to run on their Integrated Heterogeneous System (IHS) platform which were so far restricted to graphics rendering. The HSA Foundation \cite{hsafoundation} was setup to develop and define cross-vendor hardware specifications and software development tools needed to allow application software to better use this IHS architecture. This architecture provides a shared virtual address space making pointer sharing semantics possible between CPU and GPU which simplifies programming. Further a shared physical address space reduces GPU initialization time and enables several high level languages to also take advantage of the parallel processing synergistically with the CPU. Programmers can now write applications that seamlessly integrate CPUs with GPUs while benefiting from best attributes of each. Fine-grain data parallel sections in applications like garbage collection of virtual machines \cite{sumatra} and parallel stream processing like face detection, compression, encryption-decryption etc. can now use the integrated GPGPU to deliver better performance. Recent works have proposed allowing GPUs to invoke traditional operating systems paging mechanisms and hardware MMUs to fault pages. This provides the added benefit to be able to run GPU programs whose dataset sizes are not constrained by the memory size. As IHS platforms gain widespread adoption in HPC systems, improving the performance of these platforms will become of paramount importance in the near future. \cite{apu-exascale,amd-exascale1}
\par In contrast to the processing capabilities of the modern microprocessor, DRAM memory speeds have not kept pace commensurately to serve the increasing demands of the processors. This speed imbalance coupled with a limited growth in pin counts has led to the memory and bandwidth wall \cite{memory-wall,bandwidth-wall} for off-chip DRAM systems which often becomes a performance limiting factor. The advent of die-stacking technology \cite{3d-stacking} provides a way to integrate disparate silicon die of NMOS DRAM chips and CMOS logic chips with better interconnects. The implementation is accomplished either by 3D vertical stacking of DRAM chips using through-silicon vias (TSV) interconnects or horizontally/2.5D stacking on a interposer chip as depicted in Figure \ref{hsa-arch}. This allows the addition of a sizable DRAM chip close to processing cores. The onchip DRAM memory can provide anywhere from a couple of hundreds of megabytes to a few gigabytes of storage at high bandwidths of close to 400GB/s compared to the 90GB/s of DDR4 bandwidth [ref]. The better interconnect also lowers the latency of access by around 20-25\% compared to off-chip memory [ref]. Several recent proposals advocate the use of on-chip DRAM capacity as a hardware managed large last level cache for improving performance of multicore CMPs. In the context of IHS architecture, the stacked DRAM can cater to the large bandwidth requirements of throughput oriented GPUs which exploit high levels of MLP. While the latency oriented CPU applications can benefit from reduced latency of data access from the stacked DRAMCache. This also reduces energy consumed per access for the overall system in line with the goals of IHS.
\par Managing contention for shared DRAMCache in the memory hierarchy of the two heterogeneous processor architectures which have asymmetric sensitivity and demands introduces novel challenges. When a GPU kernel is launched it creates large number of concurrently running threads in lock-step in a SIMD execution model and sends large number of requests into the memory hierarchy, which creates a flood of requests that cause congestion in the memory hierarchy while the GPU is executing. The GPU can also switch execution between groups of threads to hide this memory access latency to further exploit parallelism which exacerbates the problem. This causes bottlenecks in request queues and contention for sets in the DRAMCache thus severely hampering CPU performance. Although the GPU can sustain longer memory latencies, as capabilities of execution units in GPUs of IHS chips increase, threads will be able issue larger number of memory requests, thus thrashing the system throughput. Since the GPGPU will be used intermittently for offloading parallel parts of the program, simply reserving shared resources will lead to idling and under-utilization. Also, biasing the design for throughput or latency will adversely impact the performance of the other processor. This necessitates the need for a careful design of memory system for IHS processors, that can handle the GPGPU bursty behavior as well as service requests for the CPU with a seamless/consistent latency without idling resources while delivering improved system throughput at lower energy.

To summarize, we propose to improve the system performance of IHS processors by adding a large capacity stacked DRAMCache. This first level shared memory capacity between CPU and GPU would be used as a hardware managed cache. Our organization \textit{\cachename} is aware of the asymmetric and contrasting requirements from the heterogeneous processors. \textit{\cachename} attempts to meet CPUs requirement of reduced hit times and lower miss penalties while at the same time improving hit rates for GPU to allow it to better use the DRAMCache bandwidth. In the common case when CPU is running alone \textit{\cachename} is a aggressive direct mapped DRAMCache optimized for hit times and lower miss penalty through hit/miss prediction as in \cite{alloy}. However, when GPU is running the CPU is temporally bypassed to utilize the DRAM memory using a bloom filter to identify cache misses and clean cache blocks. Secondly, The CPU is also prioritized at the DRAMCache to reduce the effect of large waiting times caused due to burst of GPU requests. Thirdly, \cachename forces spatial occupancy control by providing a pseudo associativity for GPUs to improve hit rates while still allowing CPU to obtain some benefits of caching. \cachename achieves this using a lightweight and dynamic scheme which does not impose hard partitions on the cache.

\par To the best of our knowledge, this is the first study on the interactions of IHS with a shared stacked DRAMCache. The rest of the paper is organized as follows. Section \ref{motivation} demonstrates the performance that can be gained by adding a DRAMCache to a IHS processor chip and motivates the need for architecting an asymmetry aware DRAMCache organization. We present the organization and design principles of \cachename and its describe its working in Section \ref{mechanism} . Next, in Section \ref{methodology} we describe the experimental setup and methodology followed by evaluation results in Section \ref{results}. Section \ref{related-work} presents related work in this area and Section \ref{conclusion} concludes the paper.
