\section{Conclusion} \label{conclusion}
In this work, we firstly presented a case for improvement of performance for CPU and GPU in an IHS architecture. The effects of interference due to co-running adversely impact the CPU significantly compared to the GPU. Secondly, we proposed to recover this lost performance due to co-running by the use of a large capacity stacked DRAMCache. We show that addition of such a cache can be a very good fit for each processor type to improve system performance. For this we adapt an aggressive direct mapped Alloy cache for IHS workloads. Thirdly, we improve system performance using a heterogeneity aware scheme for this DRAMCache. \cachename improves performance using heterogeneity aware spatial and temporal techniques and apply these suitably for the first time in a IHS architecture. In conclusion \cachename achieves significant improvement of about 200\% in overall system performance on an average over a baseline system with no DRAMCache and 45\% over a carefully designed but heterogeneity naive DRAMCache. This work thus shows that there are significant benefits of using a stacked DRAMCache for IHS processors, far exceeding the usefulness of the such devices in homogeneous GPUs and multi-core CPU systems.