\section{Future Work \& Conclusion} \label{conclusion}
A detailed study of energy efficiency of IHS systems with a stacked DRAM as cache is deferred to future work as this requires additional enhancements to the simulator owing to the heterogeneous nature of the cores. Intuitively as a result of the significant speedups obtained using \cachename\ the static energy dissipation of the system will be lower. The design and optimization of NoC and interconnect topology for IHS architecture is outside the purview of this work and is also deferred as a future work. We would also like to explore the design of virtual memory subsystem shared by CPU and GPU cores in an IHS architecture in the future.
\par In this work, we presented a case for performance improvement of an IHS processor by addition of a stacked DRAMCache. We quantify the effects of interference due to co-running on each processor and show that the heterogeneity adversely effects CPU performance compared to the GPU. 
We carefully design an effective DRAMCache organization for IHS processors
%and show that addition of such a cache can be a very good fit for each processor type to improve system performance.
and improve IHS performance using three simple and effective heterogeneity aware techniques - a DRAMCache scheduler, a spatial bypass and a temporal occupancy control. \cachename\ achieves significant improvement of 200\% in overall system performance on an average over a baseline system with no DRAMCache and 41\% over a heterogeneity unaware DRAMCache. This work, thus shows that there are significant benefits of using a stacked DRAMCache for IHS processors, far exceeding the usefulness of the such devices in homogeneous GPUs and multi-core CPU systems.