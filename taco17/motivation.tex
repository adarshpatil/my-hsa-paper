\section{Motivation} \label{motivation}
As mentioned in section \ref{introduction}, the large capacity provided by the stacked DRAM is in line with the working set requirements of IHS processors. Using this capacity as a hardware managed cache could provide performance gains without any application modifications. The bandwidth benefits and modestly improved latency provided by the DRAMCache help improve performance of IHS processors over on-chip SRAM caches of reasonable sizes \cite{amd-exascale1}. 
\begin{figure}[htbp]
   \centering
   \includegraphics[scale=0.6]{graphs/motivation-cpu}
   \includegraphics[scale=0.6]{graphs/motivation-gpu}
   \caption{Performance comparison of (a) CPU (b) GPU when running alone, co-run, with and w/o DRAMCache}
   \label{fig:motivation}
\end{figure}
\par In this paper, we assume a cache organization similar to Alloy cache \cite{alloy} with a block size of 128 bytes and study the problems and challenges in designing the DRAMCache for IHS architectures. We justify the design decisions in the following section. 
Each CPU core has a private L1 cache and a shared split L2 cache across the CPU cores. The GPU cores have private L1 and shared L2 cache among themselves.  
The stacked DRAMCache (of size 64MB) is the first level shared cache across CPU cores and GPU cores. In our experiments, we consider a multi-programmed workload on the CPU cores and a GPGPU application on a single CPU core and multiple CUs 
\footnote{It should be noted here that the GPU application has a CPU component and alternates execution on CPU core and the CUs (kernel execution).}. 
We refer to this workload as Heterogeneous Application (HeA), as the CPU and GPU applications are co-run. We use the terms Homogeneous Application (HoA-CPU and HoA-GPU) to refer to the cases when (multiple) CPU applications are run alone and GPU application is run alone. 
%Our study on the interference due to co-running CPU and GPU application reports performance numbers of the GPU application when the GPU kernel is running on the CUs.
Additional details relating to the experimental methodology and workloads are described in Section \ref{methodology}


\begin{figure}[htb]
   \centering
   \includegraphics[scale=0.8]{graphs/motivation-cpu-cache}
   \caption{DRAMCache CPU Access Latency \& Hit rate}
   \label{fig:motivation-cpu-cache}
\end{figure}
First, we evaluate the benefits of having a large DRAMCache for the CPU in an IHS architecture. Figure \ref{fig:motivation}(a) presents the performance improvement for CPU cores due to the addition of a stacked DRAMCache (in terms of harmonic mean of IPCs, normalized to HeA without a DRAMCache) when co-run with a GPU and when run alone.
% that is obtained by adding a stacked DRAMCache when co-run with the GPU application and also when run alone. 
As can be seen from the figure, the addition of a stacked DRAMCache improves performance of CPU applications in an IHS architecture, on an average by just 42\%. However, when the CPU runs alone with a stacked DRAMCache it achieves a 3.72x better performance than that achieved in IHS without a DRAMCache. 
%In other words, the performance gain by adding a DRAMCache is on an average 2.6x lesser in IHS. 
We observe that this huge performance gap is primarily due to the interference of GPU applications.
Although this HoA-CPU performance cannot be achieved in IHS there still exists a significant opportunity for improvement.
This gap in performance can be attributed to the unmanaged heterogeneity and interference in the DRAMCache organization.
\par We further investigate the cause of this performance gap. Figures \ref{fig:motivation-cpu-cache} presents the DRAMCache access latency experienced by a CPU request (in terms of CPU cycles) on the primary y-axis and the CPU hit rates on the secondary y-axis, while running alone and co-run with the GPU application. We find that the presence of GPU application increases the average access latency of the DRAMCache by 213\% while hit rates of the CPU are marginally impacted (only by about 4\%) when co-running.
This increase in average access latency is primarily attributed to the large
number of GPU requests flooding the DRAMCache controller when the GPU kernels are co-running with the CPU applications.
It should be noted here that, even though we use the terms CPU and GPU requests separately, they may refer to the same data. This terminology only indicates the source of the request. 
%We also note that the GPGPU application allows some amount of sharing of data within the DRAMCache between the CPU and GPU processors (i.e. constructive interference - blocks brought in by the CPU and then used by the GPU and vice versa).
\par Next, we study the impact of co-running on the GPU applications. Figure \ref{fig:motivation}(b) presents GPU performance obtained by addition of a stacked DRAMCache when co-run with the CPU and when run alone.
We observe that with the introduction of DRAMCache in an IHS, the GPU performance improves on an average by 24\%. This performance is within 10\% of its HoA GPU performance with DRAMCache. 
Thus, co-running with CPU applications has only a minor impact on the GPU performance. 
%Some cache-sensitive GPU workloads show benefits from the addition of the DRAMCache while other workloads, that do not significantly reuse cached data, show no performance benefits.
%\begin{figure}[htb]
%   \centering
%   \includegraphics[scale=0.8]{graphs/motivation-gpu-cache}
%   \caption{GPU Miss reduction with 2-way associativity}
%   \label{fig:motivation-gpu-cache}
%\end{figure}
%\par Additionally, we observe that providing associativity in DRAMCache reduces the number of GPU misses going to the relatively constricted off-chip DRAM bus. 
%Figure \ref{fig:motivation-gpu-cache} plots the reduction in number of GPU misses in log scale, by introducing 2-way associativity over our direct mapped DRAMCache. The GPU misses reduce by an average of $10^5$ for 2-way associative cache with half the number of sets and $10^6$ for a 2-way associative cache with the same number of sets.
%In comparison, the total number of CPU read requests to the DRAM are in the order of $10^7$. This considerable reduction in GPU misses leads to lower congestion for CPU requests at the DRAM. While associativity leads to only a modest improvements in hit rates for the GPU, it avoids incurring double queuing latency for all the reduced GPU misses. This corroborates the observation by other works \cite{oscar} that the number of requests from the GPU are orders of magnitude higher than CPU.
%Thus, providing some associativity can increase DRAMCache bandwidth utilization and improve the system performance.

While the above discussion indicate that the design decisions of heterogeneity aware DRAMCache are influenced by the latency-sensitive CPU applications, we emphasize the need to effectively utilize the large capacity and the higher bandwidth of DRAMCache by GPU applications. Towards this goal, we experimented with a DRAMCache of 128MB and 2-way associativity. We observe that the hit rate for GPU applications improves, on an average, by 5\%. While the improvement in hit-rate is not significant, as the number of GPU requests is very large (2 or 3 orders of magnitude higher than the CPU requests), even such a small increase in the hit rate leads to large improvements in utilization of the DRAMCache bandwidth by the GPU application. This makes a case for improving the associativity for GPU requests by introducing some pseudo-associativity in the DRAMCache, without impacting the hit-time offered by a direct-mapped tag-and-data (TAD) cache.

\par Based on the above motivation study we conclude that there are significant performance benefits that could be obtained 
with the introduction of the stacked DRAMCache for the latency-sensitive CPU applications. However, in a naive implementation
of the DRAMCache, these benefits can be lost due to interference from the co-running GPU application.  Hence it is important to carefully 
architect the DRAMCache organization to ensure CPU applications are not hampered due to this interference.
This requires the design to be aware of the heterogeneity of the applications (CPU vs GPU) and their demands on the 
memory hierarchy.  
%TACO CHANGES HERE
\par To quantify the effect of interference between CPU and GPU we further experimented with a hard partitioned DRAMCache. We partition the request queues and allocate different set of banks for CPU and GPU data. We divide the 32 banks in the DRAMCache into 20 banks (62.5\% cache size) and 12 banks (37.5\% cache size) to alleviate bank conflicts. We use a data management and coherence policy such that the data requested by the core is found in its corresponding partition of the DRAMCache. Cache misses within the partition incur memory access latency. Although such a partitioning scheme is not realistic, it gives us the ceiling for the performance that can be expected with a partitioning approach. With this setup, we experiment with allocating larger cache partition for the CPU first. The second bar in Figure \ref{fig:motivation-partition}(a) and (b) shows the performance of the CPU and GPU with such a partitioning scheme (in terms of harmonic mean of IPCs, normalized to HeA with an un-partitioned DRAMCache). We observe that the performance of the CPU reduces by mean of 5.2\% while the performance of the GPU too reduces by a mean of 2.6\%.
Next we allocate the larger DRAMCache partition to the GPU cores. The third bar in Figure \ref{fig:motivation-partition}(a) and (b) shows the performance of CPU and GPU for such a partitioning scheme respectively. We observe that the performance of the CPU and GPU cores reduces by 7.7\% and less than 0.1\%. Some of the workloads, e.g., Qg1, Qg2, and Qg5 experience considerable fall in the H-mean of CPU cores (by more than 30\%), and in none of the workloads the performance of CPU or GPU cores improves significantly.
\begin{figure}[htbp]
	\centering
	\includegraphics[scale=0.6]{graphs/motivation-cpu-partition}
	\includegraphics[scale=0.6]{graphs/motivation-gpu-partition}
	\caption{Performance comparison of (a) CPU (b) GPU when with un-partitioned D\$ and partitioned D\$}
	\label{fig:motivation-partition}
\end{figure}
Thus, hard partitioning of DRAMCache is not an effective design as it leads to under-utilization of the large capacity 
stacked DRAM, as the effects of interference due to co-running GPU application is felt only when the kernel is run on the GPU sporadically.
Further, effectively utilizing the under-utilized main memory bandwidth \cite{micro-refresh, mainak-hpca, bear}
is also important to achieve higher performance.  Lastly, the design should ensure that the CPU applications and the 
GPU application are able to utilize the large capacity of the stacked DRAM effectively to meet the working set 
requirements of the respective applications in the best possible way. In the next section we present our Heterogeneity-Aware Shared DRAMCache (\cachename) design which addresses these issues.
