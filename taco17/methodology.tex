\section{Evaluation Methodology} \label{methodology}
In this section, we describe the experimental methodology used in our evaluation. 

\par \textbf{\cachename\ Configuration:} The DRAMCache we evaluate here is a memory-side (discussed in \ref{discussion:coherence}),
first level shared cache between the 2 split cache hierarchies of CPUs and GPUs. Access to memory is interleaved across multiple memory controllers. To avoid a biased interleaving due to uniformly strided address patterns, a basic XOR-based address hashing mechanism is employed. Each memory controller manages a 4GB DDR3 memory device and a 32MB stacked DRAM vault. The stacked DRAM vault caches data from the corresponding 4GB memory device that the controller is responsible for. This setup ensures that there are no cross bus requests between controllers. Since our simulation setup has the limitations of having a 1-to-1 channel mapping between a stacked DRAM vault and a off-chip DRAM channel, our model does not provide sufficient channel level parallelism as is essential in a stacked DRAM device. To circumvent this limitation we increase the number of layers (ranks) to provide higher amount of parallelism in our stacked DRAM. However, the the bank capacity is retained as would be in a standard stacked DRAM. The DRAMCache is non-inclusive  \cite{coherence-dramcache} of the L2 caches above it and uses a write no-allocate policy.
\par The NoC is modelled as a hierarchical cross-bar topology. One cross bar connects the CPU and caches to each other and another cross bar connects the GPU to the caches. A third cross bar connects the Last Level Caches of the CPU and GPU to the DRAMCache memory controller. The DRAMCache controller does the request scheduling and dispatches commands to the stacked DRAM devices. The DRAMCache controller and off-chip DRAM controller are connected by a point-to-point link. The DRAMCache controller forwards request to the off-chip DRAM controller if necessary.

\begin{table}[h]
  \scriptsize
  \centering
  
  \begin{tabular}{{@{}ll@{}}}
    \toprule
    CPU Core 	& five 4-wide OoO x86 cores @2.5 GHz \\
    \midrule
    CPU Caches 	& 32KB 8-way split I/D private L1 Cache, 1MB 8-way shared split L2 Cache, 128B lines \\
    \midrule
    GPU Core 	& eight Fermi SMs@700MHz, 2x2 GTO warp sched \\
    			& upto 32 threadblocks, 64 warps of 32 threads, 64K registers, 96KB scratch memory \\
    \midrule
    GPU Caches 	& 64KB 4-way private L1 cache, 512KB, 16-way assoc L2 Coherent Cache \\
    \midrule
    Stacked     & 2 vaults, HMC\_2500\_x64, 2KB Page, \ $t_{CL}$-$t_{RCD}$-$t_{RP}$-$t_{RAS}$=9.9ns-10.2ns-7.7ns-21.6ns\\
	DRAM		& 8 layers/vault, 4 banks/layer, 64 byte burst size, Peak bandwidth 40GB/s \\
				& Refresh related: $t_{REFI}$=3.9us $t_{RFC}$=59us \\
    \midrule
    Off-chip 	& 2 channels, DDR3\_1600\_x64, 1KB page, \ $t_{CL}$-$t_{RCD}$-$t_{RP}$-$t_{RAS}$=13.75ns-13.75ns-13.75ns-35ns\\
    DRAM		& 1 rank/channel, 8 banks/rank, 64 byte burst size, Peak bandwidth 25GB/s \\
    			& Refresh related: $t_{REFI}$=7.8us $t_{RFC}$=260us \\
    			  
    \bottomrule
  \end{tabular}
  \caption{Configuration of the simulated system}
  \label{configuration}
  %\vspace{-2em}
\end{table}

\par \textbf{Simulator:} We evaluate the performance of the DRAMCache over a IHS Processor using a modified version of the cycle accurate simulator gem5-gpu \cite{gem5-gpu} which is configured to simulate cache coherent unified CPUs and GPUs. 
Our IHS consists of 5 CPU cores (4-wide OoO cores, operating at 2.5GHz) with a 32KB private L1 Cache (split I/D), and a 1MB shared L2 Cache (shared across all CPU cores). The IHS also consists of 8 Fermi-like compute units, operating at 700MHz with a private 64KB L1 and 512KB shared L2 cache (shared across all CUs). The details of the IHS configuration are given in Table \ref{configuration}. The private L1 of CUs are non-inclusive of the shared GPU L2 cache and can hold stale data. However, GPU L2 cache is coherent with all levels of the CPU hierarchy. The CPU caches and GPU L2 cache are kept coherent in the simulator. Table \ref{configuration} provides the simulator setup details. 
Our simulator also respects all significant timing and functional parameters for the stacked DRAMCache (including refresh, data bus, request queues, scheduling algorithms, command signalling and clock frequencies) using the DRAMCtrl \cite{dramctrl} model. We have also faithfully modelled  a Fill Queue \cite{dca} for fill requests to insert data into the cache on the return path from main memory. We also model MSHR and WriteBuffers with their associated latencies to realize the precise working of caches. 
\par Further, our baseline setup with a naive DRAMCache is equivalent to the on-chip shared caches in \cite{oscar} without the NoC related improvements, which is orthogonal to the ideas presented in this work.

\par \textbf{Workloads:} Applications having high and medium memory intensive behaviours from SPEC CPU2006 suite \cite{spec2006} were chosen to form a multi-programmed workload. We use the Rodinia benchmark suite \cite{rodinia} to represent GPU applications that offload kernel computation to a GPU CUs. These Rodinia applications are modified to elide the \textit{memcpy} api calls so as to run with unified virtual and physical address spaces. Combinations of quad-core multi-programmed workloads were coupled with a full Rodinia benchmark to create a representative mix of applications that would run in an IHS system. These composite workloads embody different levels of total memory intensity produced by the CPU and GPU cores. Our simulation of the simultaneous activity on both CPU and GPU cores using co-running multi-programmed workload along with Rodinia-nocopy benchmarks allows us to demonstrate the interleaving of memory traffic at the stacked DRAM and off-chip DRAM. We also measure the footprint of these workloads in terms of number of unique 128B cache blocks accessed at the DRAMCache. The memory footprints range from 70MB to 650MB for quad-core CPU workloads and from 5.5MB to 135MB for the GPU application. 
%TACO REVIEW CHANGES HERE
The smaller footprints obligates us to use a smaller stacked DRAMCache capacity of 64MB for all workloads and configurations to make pertinent observations.


\begin{comment}
\begin{table}[h!]
  \footnotesize
  \centering
  \begin{tabular}{{|l|l|l||l|l|l|}}
    \hline
    \textbf{WID} & \textbf{Multi-program SPEC2006} & \textbf{Rodinia} & \textbf{WID} & \textbf{Multi-program SPEC2006} & \textbf{Rodinia} \\
    \hline
    Qg1& cactusADM;gcc;bzip2;sphinx3 & needle & Qg8 & milc;libquantum;gobmk;leslie3d & gaussian \\
    \hline
    Qg2 & astar;mcf;gcc;bzip2 & needle & Qg9 & astar;milc;gcc;leslie3d & hotspot \\
    \hline
    Qg3 & gcc;libquantum;leslie3d;bwaves & needle & Qg10 & gcc;gobmk;leslie3d;sphinx3 & hotspot\\
    \hline
    Qg4 & astar;soplex;cactusADM;libquantum & k-means & Qg11 & astar;cactusADM;libquantum;sphinx3 & srad \\
    \hline
    Qg5 & milc;mcf;libquantum;bzip2 & k-means & Qg12 & astar;mcf;gobmk;sphinx3 & stream\\&&&&&cluster\\
    \hline
    Qg6 & bzip2;gobmk;hmmer;sphinx3 & k-means & Qg13 & astar;cactusADM;libquantum;sphinx3 & lud\\
    \hline
    Qg7 & soplex;milc;cactusADM;libquantum & gaussian & & &\\
    \hline    
  \end{tabular}
  \caption{Workloads (WID: Workload ID)}
  \label{workloads}
  %\vspace{-2em}
\end{table}
\end{comment}

\begin{table}[h!]
	\footnotesize
	\centering
	\begin{tabular}{{|l|l|l|}}
		\hline
		\textbf{Name} & \textbf{Multi-program SPEC2006} & \textbf{Rodinia}\\
		\hline
		Qg1& cactusADM;gcc;bzip2;sphinx3 & needle\\
		\hline
		Qg2 & astar;mcf;gcc;bzip2 & needle\\
		\hline
		Qg3 & gcc;libquantum;leslie3d;bwaves & needle\\
		\hline
		Qg4 & astar;soplex;cactusADM;libquantum & k-means\\
		\hline
		Qg5 & milc;mcf;libquantum;bzip2 & k-means\\
		\hline
		Qg6 & bzip2;gobmk;hmmer;sphinx3 & k-means\\
		\hline
		Qg7 & soplex;milc;cactusADM;libquantum & gaussian\\
		\hline
		Qg8 & milc;libquantum;gobmk;leslie3d & gaussian\\
		\hline
		Qg9 & astar;milc;gcc;leslie3d & hotspot\\
		\hline
		Qg10 & gcc;gobmk;leslie3d;sphinx3 & hotspot\\
		\hline
		Qg11 & astar;cactusADM;libquantum;sphinx3 & srad\\
		\hline
		Qg12 & astar;mcf;gobmk;sphinx3 & streamcluster\\
		\hline
		Qg13 & astar;cactusADM;libquantum;sphinx3 & lud\\
		\hline
	\end{tabular}
	\caption{Workloads}
	\label{workloads}
	%\vspace{-2em}
\end{table}

\par \textbf{Simulation Methodology:} We fast-forward the initialization phase of each workloads up until just before the launch of the first kernel of the GPU program. We ensure that each core is fast-forwarded by atleast 2 Billion instructions and in total each quad-core workload runs 20 billion instructions on average in the fast-forward phase. This is accomplished by adding a pause phase to the Rodinia benchmarks for the duration until the initialization quota of the SPEC programs is complete. We then warm the cache until the fastest core completes 250 million instructions. During the warm-up phase the GPU program is not executed. Timing simulations were then run for atleast 250 million instructions for each CPU core, resulting in a total of more than 1 Billion instructions across all the CPU cores. As is the norm, when a core finishes its quota of 250 million instructions, it continues to execute until all the cores have completed. \\
The Rodinia application uses an extra 5\textsuperscript{th} CPU core and offloads to the integrated GPU. These applications are modified to execute in a \textit{conditioned loop} such that there is no corruption of data structures in the program. The \textit{conditioned loop} is run infinitely and represents the region of interest (ROI) of the Rodinia benchmark. This region includes sections of CPU activity and GPU offloads in the execution, as is typical of a HSA program which will exhibit an interleave of offloaded regions and serial CPU sections. However, only the performance statistics for the first execution of the \textit{conditioned loop} in the program are considered. 
This is done as the first loop represents the true run of the GPU kernel. The number of \textit{conditioned loops} executed depends on the length of the simulation and the IPC achieved by the GPU. Also, Subsequent loops can achieve better hit rates in cache due to the data fetched by the earlier loops thus not corroborating with the true performance achieved.
In cases where the ROI is longer than the complete run of the CPU workloads, the statistics for the last completed kernel are used. 

\par \textbf{Coherence and Memory Consistency:} \label{discussion:coherence}
The DRAMCache evaluated here is a memory-side cache \cite{primer-coherence-consistency, mainak-hpca, skylake}.  These caches are outside the coherence domain. These caches do not require to add additional states to the coherence protocol and do not need to be snooped separately. They are logically just in front of the memory and serve to reduce the average latency of memory accesses and increase the memory's effective bandwidth.  
\par The consistency model followed for each type of cores (CPU cores and GPU cores) is different and is dictated by the corresponding processor architecture i.e. strong memory consistency for the CPU and weak/release consistency for GPU. Cache line evictions and explicit fence operations flush data from L2 SRAM cache to memory and could be cached in the DRAMCache. Subsequent requests to memory for that cache line first look up the DRAMCache (or bypass in case of \bypassname) before sending the request to the off-chip DRAM. Further discussion about cache coherence and memory consistency models for IHS architecture is beyond the scope of this work.

\par \textbf{Performance Metric:} We report performance of each processor using the intuitive metric of harmonic mean of IPC for the CPU cores and GPU CUs which is defined as 
{\small
\begin{equation*}
H\textnormal{-}MEAN_{CPU} = \frac{n_{cpu}}{\sum_{i=1}^{n_{cpu}} \frac{1}{IPC_{i}^{CPU}}}, \hspace{0.2cm} H\textnormal{-}MEAN_{GPU} = \frac{n_{gpu}}{\sum_{i=1}^{n_{gpu}} \frac{1}{IPC_{i}^{GPU}}} 
\end{equation*}
}
where $IPC_i^{CPU}$ and $IPC_i^{GPU}$ are the instructions per cycle achieved by the $i^{th}$ CPU core and $i^{th}$ GPU CU respectively.
In this work, we report CPU and GPU $H\textnormal{-}MEAN$ (and the improvement in them) independently. This is done to identify the impact of CPU and GPU applications on each other, as well as to understand the impact of the proposed modifications on each of the application type. We report combined system performance using the combined H-MEAN of all CPU and GPU cores, as in 
{\small
\begin{equation*}
H\textnormal{-}MEAN_{IHS} = \frac{n_{cpu}+ n_{gpu}}{\sum_{i=1}^{n_{cpu}} \frac{1}{IPC_{i}^{CPU}} + \sum_{i=1}^{n_{gpu}} \frac{1}{IPC_{i}^{GPU}}} 
\end{equation*}
}
\par Last we also report combined system performance using the weighted speedup metric \cite{weighted-speedup} which is defined as
{\small
\begin{equation*}
Weighted Speedup = \sum_{i=1}^{n_{cpu}} \frac{IPC_i^{CPU_{IHS}}}{IPC_i^{SP}} + \sum_{i=1}^{n_{gpu}} \frac{IPC_i^{GPU_{IHS}}}{IPC_i^{GPU}}
\end{equation*}
}
where ${IPC_i^{CPU_{IHS}}}$ and $IPC_i^{GPU_{IHS}}$ denote the IPC achieved by the $i^{th}$ CPU core and the $i^{th}$ GPU CU when running in a IHS setup respectively; whereas $IPC_i^{SP}$ and $IPC^{GPU}$ denote the same when $i^{th}$ CPU core and the $i^{th}$ GPU CU are running standalone respectively. 
