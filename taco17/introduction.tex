\section{Introduction}\label{introduction}



\par The remarkable advances in computing power of the modern microprocessor over the last few decades can predominantly be attributed to shrinking of feature sizes. Miniaturization of transistors has allowed addition of specialized on-chip hardware circuitry for acceleration units. 
%A study in 2010 by Koomey et. al \cite{koomey} found that the amount of computation that could be done per unit of energy doubled about every 18 months. However, to reach exascale and beyond requires a thousand fold decrease in energy consumed per flop computed.
Graphics processing units (GPUs) have evolved from being fixed-function pipelines and are being used to accelerate data parallel code for general purpose (GP) applications. Compared with multi-core CPUs, GPGPUs offer the potential for better performance at lower energy. Traditionally these discrete processors have had their own independent memory systems. To take advantage of the discrete GPUs, the CPU must copy data to the GPUs memory and back. This data movement is wasteful and expends energy while also adding latency as the transfer happens over a slower PCIe bus. The separate address spaces and complex programming models that need to manage two sets of data further impede expansion of the workloads that benefit from GPGPU computing. 

\par Modern processor chips have heterogeneous processors which integrate a lower capacity GPU on-die allowing for graphics rendering only. In view of the widespread use of the GPU for general purpose applications, processor manufacturers including AMD\cite{amd-apu}, Intel\cite{inteliris}, and NVIDIA\cite{denver} are beginning to allow general purpose OpenCL\cite{opencl}/CUDA\cite{cuda} programs to run on their Integrated Heterogeneous System (IHS) platform which were so far restricted only to graphics rendering. The HSA Foundation \cite{hsafoundation} was setup to develop and define cross-vendor hardware specifications and software development tools needed to allow application software to better use this IHS architecture. 
The IHS architecture, as illustrated in Figure \ref{fig:hsa-arch}(a), consists of multiple CPU cores with private L1 and shared L2 caches along with multiple GPU cores or Compute Units (CUs) with private L1 and shared L2 cache. These caches (except GPU L1 caches) are kept coherent with a heterogeneous MOESI-like coherence protocol. The Network-on-chip (NoC) connects the cores with the caches and the memory controllers, while the memory itself is off-chip and connected via memory channels.

\begin{figure}[!htb]
	\centering
	\def\svgwidth{0.48\columnwidth}
	\input{images/arch.pdf_tex}
	\def\svgwidth{0.5\columnwidth}
	\input{images/stackedDRAM.pdf_tex}
	\caption{(a)Architecture of an Integrated Heterogeneous System (b)Proposed DRAM Stacking for IHS}
	\label{fig:hsa-arch}
\end{figure}

\par The IHS processors have a cache coherent interconnect and support shared virtual address space making pointer sharing semantics possible between CPU and GPU which simplifies programming and allows fine-grain parallel processing.
%Further, a shared physical address space and the cache coherent interconnect reduces GPU initialization time and enables several high level languages \cite{sumatra,julia} to also take advantage of the parallel processing synergistically with the CPU. Programmers can now write applications that seamlessly integrate CPUs with GPUs while benefiting from best attributes of each. Fine-grain parallel stream processing like face detection, compression, encryption-decryption etc. can now use the integrated GPGPU to deliver better performance. Recent works have proposed allowing GPUs to invoke traditional operating systems paging mechanisms and hardware MMUs to fault pages \cite{tlb-translation}. This provides the added benefit to be able to run GPU programs whose dataset sizes are not constrained by the memory size. As IHS platforms gain widespread adoption in HPC systems, improving the performance of these platforms will become of paramount importance in the near future. \cite{apu-exascale,amd-exascale1}

%\par In contrast to the processing capabilities of the modern microprocessor, DRAM memory speeds have not kept pace commensurately to serve the increasing demands of the processors. This speed imbalance coupled with a limited growth in pin counts has led to the memory and bandwidth wall \cite{memory-wall,bandwidth-wall} for off-chip DRAM systems which often becomes a performance limiting factor. 
\par The advent of die-stacking technology \cite{3d-stacking} provides a way to integrate disparate silicon die of NMOS DRAM chips and CMOS logic chips with better interconnects. The implementation is accomplished either by 3D vertical stacking of DRAM chips using through-silicon vias (TSV) interconnects or horizontally/2.5D stacking on an interposer chip as depicted in Figure \ref{fig:hsa-arch}(b). This allows the addition of a sizeable DRAM chip of capacities ranging from a couple of hundreds of megabytes to a few gigabytes close to the processing cores. These stacked DRAM devices provide high bandwidths of close to 400GB/s compared to the 90GB/s of DDR4 bandwidth \cite{xeonphi}. The better interconnect also lowers the latency of access compared to off-chip memory \cite{alloy}. 
\par SRAM has traditionally been used for smaller caches closer to the processing elements because of the need for fast access times.
However, modern last level SRAM caches typically have capacities between 2 to 8 MB \cite{skylake} and this is often not large enough to contain the working set of IHS workloads.
Moreover, GPUs have large number of concurrent threads and their access patterns thrash the LLC, removing useful information from the cache due to its small size. 
Non volatile STT-RAM caches have been proposed in recent literature as replacement for on-chip SRAM caches \cite{oscar}. STT-RAM caches provide about 4X the capacity of SRAM \cite{oscar} (in the order of 32MB) for the same area due to its higher cell density.
Stacked DRAMs offer much high capacities than its SRAM and STT-RAM counterparts by one to two orders of magnitude \cite{3d-stacked}.
\par Further, stacked DRAMCaches are attractive because they provide much larger bandwidth than that of an on-chip SRAM and STT RAM caches.
This is especially compelling as IHS architecture pack multiple CPU and GPU cores have higher memory bandwidth demands and these inter-die vias and interconnects in the 3D chips can supply these data hungry processors.
Stacked DRAMs are not constrained by the bandwidth wall problem as the TSV interface between the DRAM stacks and the logic chip scales with the surface area of the chip rather than with the number of pins at its perimeter. This allows direct, parallel access to the DRAM modules than accessing the memory external to the chip.
\par Stacked DRAM can also be used in conjunction with the on-chip SRAM/STT-RAM caches to augment the memory with higher capacities and bandwidth. We expect the key observations of the heterogeneous nature of the requests and the inferences thereof to be similar even with the shared on-chip caches in IHS architectures.
\par Several recent proposals advocate the use of on-chip DRAM capacity as a hardware managed last level cache for improving performance of multicore CMPs \cite{alloy,bimodal,loh-hill,atcache}. In the context of IHS architecture, the stacked DRAM can cater to the large bandwidth requirements of throughput-oriented GPUs while the latency-sensitive CPU applications can benefit from reduced latency of data access. 

\begin{comment}
\begin{figure}[!htb]
	\centering
	\def\svgwidth{0.5\columnwidth}
	\input{stackedDRAM.pdf_tex}
	\caption{Proposed DRAM Stacking for IHS}
	\label{fig:stackdram}
\end{figure}
\end{comment}

%This also reduces energy consumed per access for the overall system, in line with the goals of IHS.
\par Managing contention for shared DRAMCache in the memory hierarchy of the two heterogeneous processor architectures which have asymmetric sensitivity and demands, however, introduces novel challenges. When a GPU kernel is launched it creates large number of concurrent threads which run in lock-step SIMD execution model and sends a large number of requests into the memory hierarchy causing congestion. This causes bottlenecks in request queues at the DRAMCache thus severely hampering CPU performance. Further, GPUs are designed to tolerate longer memory latencies while the memory hierarchy of typical CPUs have been designed to optimize memory access latencies for CPUs. 
Thus latency-sensitive CPU applications would benefit from a DRAMCache design that offers lower hit-time and hence a a lower miss penalty for the previous level caches (L2 cache). On the other hand, the GPU cores would benefit from higher bandwidth and higher hit-rates in DRAMCache even at the cost of higher hit-time. This necessitates the need for a careful design of memory system for IHS processors, that can handle the GPGPU bursty behaviour as well as service requests for the CPU with a consistent latency without idling resources while delivering improved system throughput at lower energy.
\par To improve the performance of IHS systems we introduce a large capacity stacked DRAMCache, used as a hardware managed cache, as the first level shared resource in the memory hierarchies of CPU and GPU cores. Our organization, \cachename, is aware of the asymmetric and contrasting requirements from the heterogeneous processors. \cachename\ attempts to meet CPUs requirement of reduced hit times and lower miss penalties while at the same time improving hit rates for GPU to allow it to better use the DRAMCache bandwidth. In the common case when CPU is running alone, \cachename\ is an aggressive direct mapped DRAMCache optimized for hit times and lower miss penalty through hit/miss prediction as in \cite{alloy}. However, when GPU is active we propose (i) \prioname\ - a DRAMCache controller scheduling algorithm to prioritize scheduling of CPU requests in the queues to reduce the large waiting time caused due to burst of GPU requests (ii) \bypassname\ - a mechanism to allow some of the CPU requests (access requests that are clean data in DRAMCache or data that is currently not in DRAMCache) to be temporally bypassed to utilize the under-utilized off-chip memory bandwidth
(iii) \chaining\ - a technique to force spatial occupancy control by providing a pseudo associativity for GPUs to improve hit rates while still allowing certain guaranteed occupancy for CPU lines. \cachename\ achieves these goals using a lightweight and dynamic scheme which does not impose hard partitions on the shared DRAMCache. Using detailed simulations, we show that the proposed optimizations improve overall system performance, on an average by 41\% over a carefully designed, but heterogeneity unaware DRAMCache.
While the techniques proposed in the paper are simple and proposed in other contexts, deploying them in the DRAMCache helps to effectively address the disparate demands of IHS processors and improve performance of both CPU and GPU cores by 211\% and 20\% respectively, over a IHS processor with no DRAMCache.

\par While there have been studies and proposals to share the on-chip last-level SRAM caches \cite{helm,tap} and non-volatile STT-RAM caches \cite{oscar} in an IHS architectures previously, to the best of our knowledge this is the first study on the interactions of such an architecture with a stacked DRAMCache focusing on shared cache management issues.
%Our proposal \cachename addresses problems unique to stacked DRAMs and applies optimizations accordingly to address these challenges.
The rest of the paper is organized as follows: Section \ref{motivation} demonstrates the performance that can be gained by adding a DRAMCache to a IHS processor chip and motivates the need for architecting a heterogeneity aware DRAMCache organization. We present the design principles, organization and the working of \cachename\ in Section \ref{design+mechanism} . Next, in Section \ref{methodology} we describe the experimental setup and methodology. This is followed by evaluation and results in Section \ref{results}. Section \ref{related-work} presents related work in this area and Section \ref{conclusion} concludes the paper.
