\usetikzlibrary{shapes,arrows}
\usetikzlibrary{backgrounds}
\usetikzlibrary{matrix, positioning, fit}
\usetikzlibrary{patterns}
\definecolor{forestgreen}{rgb}{0.0, 0.5, 0.0}

\tikzstyle{smallcircle} =[fill=black!100, text=white, circle, inner sep=1pt, minimum size=0.1em]
\tikzstyle{ourcircle} = [draw, semicircle, inner sep=0pt,minimum size=2pt]
\tikzstyle{dullblock} = [draw, fill=black!20, circle, minimum height=2em, minimum width=2em]
\tikzstyle{block} = [draw=black, rounded corners, thick, line width=0.3mm, rectangle, minimum height=10em, minimum width=7em]
\tikzstyle{blocksmall} = [draw=black, thick, line width=0.5mm, rectangle, minimum height=6em, minimum width=3em, fill=white]
\tikzstyle{group} = [draw=black, line width=0.3mm, rectangle, minimum height=2em, minimum width=2em]
\tikzstyle{textblock} = [draw, fill=black!20, rectangle, rounded corners]
\tikzstyle{rectblock} = [draw, rectangle, minimum height=1.3em]
\tikzstyle{plain} = []
\tikzstyle{decisionblock} = [draw, diamond, fill=black!20]
\tikzstyle{input} = [draw, thick, fill=blue!20, circle, minimum size=1pt]
\tikzstyle{output} = [draw, fill=blue!20, circle, minimum size=1pt]
\tikzstyle{pinstyle} = [pin edge={to-,thin,black}]
\tikzset{toprule/.style={%
        execute at end cell={%
            \draw [line cap=rect,#1] (\tikzmatrixname-\the\pgfmatrixcurrentrow-\the\pgfmatrixcurrentcolumn.north west) -- (\tikzmatrixname-\the\pgfmatrixcurrentrow-\the\pgfmatrixcurrentcolumn.north east);%
        }
    },
    bottomrule/.style={%
        execute at end cell={%
            \draw [line cap=rect,#1] (\tikzmatrixname-\the\pgfmatrixcurrentrow-\the\pgfmatrixcurrentcolumn.south west) -- (\tikzmatrixname-\the\pgfmatrixcurrentrow-\the\pgfmatrixcurrentcolumn.south east);%
        }
    }
}


\newcommand{\bloom}[0]{
    \begin{tikzpicture}[auto, >=latex']
        % Blocks
        \node [textblock, thick, minimum height=5em] (bye){\begin{varwidth}{4cm} \centering{\footnotesize{B\\y\\E\\}} \end{varwidth}};
        \node [textblock, thick, minimum width=6em, minimum height=4em, right = 1.5cm of bye, yshift=2.2cm] (dramcache){\begin{varwidth}{4cm} \centering{\footnotesize{DRAM\\Cache}} \end{varwidth}};
        \node [textblock, thick, left = 0.01cm of dramcache, xshift=1.5em] (pred){\begin{varwidth}{4cm} \centering{\tiny{P\\R\\E\\D\\}} \end{varwidth}};
        \node [textblock, thick, left = 0.01cm of dramcache] (queue){\begin{varwidth}{4cm} \centering{\footnotesize{I I I I}} \end{varwidth}};
        \node [textblock, thick, minimum width=4em, right = 5cm of bye, minimum height=7em, yshift=-0.5cm] (dram){\begin{varwidth}{4cm} \centering{\footnotesize{Off-Chip\\DRAM\\Memory}} \end{varwidth}};

        % Arrows
        \draw [<-,very thick] (bye.west) to [out=180,in=0] ++(-0.6,0);
        \draw [->, very thick] (dramcache.east) to [out=0,in=180] ++(2.1,0) to (dram.north);
        \draw [->, very thick] ([yshift=1em]dramcache.east) to [out=0,in=180] ++(3.5,0);
        \draw [->, very thick] (dram.east) to [out=0,in=180] ++(1.1,0);
        \draw [<-, very thick] (queue.south) to ++(0.0,-0.85) to [out=180,in=0] ++(-1.3,0) to (bye.north);
        \draw [->, very thick] (dramcache.south) to [out=-90,in=90] ++(0,-0.8) to [out=180,in=0] ++(-2.6,0);
        \draw [->, very thick] ([yshift=2.5em] dram.west) to ++(-1.6,0) to ([xshift=2.3em] dramcache.south);
        \draw [<->, very thick] (bye.south) to [out=-90,in=90] ++(0,-0.5) to [out=0,in=180] ++(5.25,0);
        \draw [->, very thick] (bye.east) to [out=0,in=180] ++(5,0);
        \draw [<-, very thick] (queue.west) to ++(-0.45,0.45) to ++(-1,0);
        \draw [<-, very thick] (queue.west) to ++(-0.45,-0.45) to ++(-1,0);
        \draw [-, very thick, xshift=3.2cm] (0,1.5) to (0,0.16);
        \draw [-, very thick, xshift=3.2cm, right=90, looseness=3] (-0.07,0.067) to (0,-0.15);
        \draw [->, very thick, xshift=3.2cm] (0.01,-0.13) to (0,-1.4);
        \draw [->, dashed] (pred.east) to [out=0,in=180] ++(1.65,0);

        % labels
        \node [plain, left=0.01cm of bye, minimum height=1em, minimum width=1em] (cpurd) {\begin{varwidth}{4cm}\scriptsize{CPU\\ \\Rd}\end{varwidth}};
        \node [plain, above=1.4cm of dram, minimum height=1em, minimum width=1em, xshift=-2.8em, yshift=-1.8em] (cachemiss) {\begin{varwidth}{4cm}\scriptsize{Cache Miss/\\Pred Off-chip acess}\end{varwidth}};
        \node [plain, left=1.2cm of dram, minimum height=1em, minimum width=1em, yshift=-3em] (dirtyevict) {\begin{varwidth}{4cm}\scriptsize{Dirty Eviction}\end{varwidth}};
        \node [plain, left=0.4cm of dramcache, minimum height=1em, minimum width=1em, yshift=-1.44em, xshift=-2em] (WrReq) {\begin{varwidth}{4cm}\scriptsize{CPU Wr\\ Req}\end{varwidth}};
        \node [plain, left=0.3cm of dramcache, minimum height=1em, minimum width=1em, yshift=1.2em, xshift=-1.8em] (GpuReq) {\begin{varwidth}{4cm}\scriptsize{GPU Rd/Wr \\Req}\end{varwidth}};
        \node [plain, right=0.2cm of bye, minimum height=1em, minimum width=1em, yshift=1.25cm, xshift=-0.53cm] (maydir) {\begin{varwidth}{4cm}\scriptsize{Maybe Dirty}\end{varwidth}};
        \node [plain, right=0.2cm of bye, minimum height=1em, minimum width=1em, yshift=0.1cm, xshift=1cm] (notdir) {\begin{varwidth}{4cm}\scriptsize{Not Dirty}\end{varwidth}};
        \node [plain, below=0.15cm of maydir, minimum height=1em, minimum width=1em, xshift=1cm, yshift=0.2cm] (wrhit) {\begin{varwidth}{4cm}\scriptsize{Write Hit}\end{varwidth}};
        \node [plain, left=0.01cm of dram, minimum height=1em, minimum width=1em, yshift=3em] (fill) {\begin{varwidth}{4cm}\scriptsize{GPU Fill}\end{varwidth}};
        \node [plain, above=0.01pt of queue, minimum height=1em, minimum width=1em, yshift=-0.3em] (reqq) {\begin{varwidth}{4cm}\scriptsize{Req Q}\end{varwidth}};
        \node [plain, right=0.01pt of dram, minimum height=1em, minimum width=1em] (cpubypass) {\begin{varwidth}{4cm}\scriptsize{CPU Resp\\Bypass}\end{varwidth}};
        \node [plain, right=1.5cm of dramcache, minimum height=1em, minimum width=1em, yshift=1em] (datadram) {\begin{varwidth}{4cm}\scriptsize{Resp from\\ DRAMCache}\end{varwidth}};


    \end{tikzpicture}

}

\newcommand{\chainaccess}[0]{
    \begin{tikzpicture}[auto, >=latex']
        % We start by placing the blocks
        \node [textblock, thick, minimum width=6em] (act){\begin{varwidth}{4cm} \centering{\footnotesize{DRAM $\$$ \\Row Act.}} \end{varwidth}};
        \node [textblock, thick, right = 0.5cm of act, minimum width=4em] (burst) {\begin{varwidth}{4cm} \centering{\footnotesize{TAD \\Burst}} \end{varwidth}};
        \node [decisionblock, thick, right = 0.5cm of burst] (tagmatch) {\begin{varwidth}{4cm} \centering{\scriptsize{Tag\\Match}} \end{varwidth}};
        \node [textblock, thick, below = 0.5cm of tagmatch] (chaintable) {\begin{varwidth}{4cm} \centering{\footnotesize{Chain\\Offset}} \end{varwidth}};
        \node [decisionblock, thick, right = 1.8cm of tagmatch.south] (clean) {\begin{varwidth}{4cm} \centering{\scriptsize{Clean?}} \end{varwidth}};
        \node [decisionblock, thick, right = 2.1cm of tagmatch, yshift=1.5em] (pamreturn) {\begin{varwidth}{4cm} \centering{\scriptsize{PAM \\return?}} \end{varwidth}};
        \node [textblock, thick, right = 1.5cm of pamreturn.south, minimum width=4em, yshift=-1em] (chaintad) {\begin{varwidth}{4cm} \centering{\footnotesize{Chain\\TAD}} \end{varwidth}};
        \node [decisionblock, thick, right = 0.5cm of chaintad] (chainmatch) {\begin{varwidth}{4cm} \centering{\scriptsize{Tag\\Match}} \end{varwidth}};
        \node [textblock, thick, right = 0.5cm of chainmatch, minimum height=7em, yshift=1.5em] (mem) {\begin{varwidth}{4cm} \centering{\footnotesize{Off-chip\\DRAM\\Memory}} \end{varwidth}};

        % Arrows
        \draw [->,very thick] (act) -- (burst);
        \draw [->,very thick, red] (chaintable.east) -- (clean);
        \draw [->,very thick] (burst) -- (tagmatch);
        \draw [->,very thick] (tagmatch.south) -- (chaintable.north);
        \draw [->,very thick] (tagmatch.east) to [out=0,in=180] ++(0.5,0) to [out=90,in=-90] ++(0,1.5);
        \draw [->,very thick, red] (clean.north) to [out=90,in=-90] ++(0,0.65) to (pamreturn.west);
        \draw [->,very thick, red] (pamreturn.south)to [out=-90,in=90] ++(0,-0.3) to [out=0,in=180] ++(1.5,0);
        \draw [->,very thick, red] (pamreturn.east) to [out=0,in=180] ++(0.5,0) to [out=90,in=-90] ++(0,0.5);
        \draw [->,very thick, red] (clean.east) to [out=0,in=180] ++(2.2,0);
%        \draw [->,very thick] (chaintable.south) to [out=0,in=180] ++(9.65,0) to (mem.south);
        \draw [->,very thick] (chainmatch.east) to [out=0,in=180] ++(0.6,0);
        \draw [->,very thick] (chaintad.east) -- (chainmatch);
        \draw [->,very thick] (chainmatch.north) to [out=90,in=-90] ++(0,1);
        \draw [->,very thick] (chaintable.east) to [out=0,in=180] ++(5.6,0) to (chaintad.south);
        \draw [->,very thick] (chaintable.east) to [out=0,in=180] ++(0.3,-0.4) to [out=0,in=180] ++(9.35,0) to (mem.south);

        % Groups
        \node [blocksmall, draw=white, above = 1cm of clean, minimum height=1em, minimum width=0.2em, xshift=-0.7em, yshift=0.4em] (lblcpu){\textbf{CPU}};
        \node[group, dashed, inner sep = 1pt, line width=0.3mm, draw=red, fit=(clean)(pamreturn)] (cpugroup) {};

        % Labels
        \node [plain, right=0.01em of tagmatch, minimum height=1em, minimum width=1em, yshift=1em, xshift=-0.5em] (lbl1) {\footnotesize{\emph{Hit}}};
        \node [plain, below=0.01em of tagmatch, minimum height=1em, minimum width=1em, xshift=-1.2em] (lbl2) {\footnotesize{\emph{Miss}}};
        \node [plain, below=0.52cm of tagmatch, minimum height=1em, minimum width=1em, xshift=4em] (lbl3) {\footnotesize{\emph{Yes}}};
        \node [plain, below=0.1cm of lbl3, minimum height=1em, minimum width=1em, yshift=0.2em] (lbl4) {\footnotesize{\emph{No}}};
        \node [plain, above=0.01cm of lbl3, minimum height=1em, minimum width=1em, red] (lb15) {\footnotesize{\emph{Yes}}};
        \node [plain, right=0.01em of chainmatch, minimum height=1em, minimum width=1em, yshift=-0.5em, xshift=-0.65em] (lbl5) {\footnotesize{\emph{Miss}}};
        \node [plain, above=3em of lbl5, minimum height=1em, minimum width=1em, yshift=-0.5em, xshift=-2em] (lbl8) {\footnotesize{\emph{Hit}}};
        \node [plain, right=0.01em of clean, minimum height=1em, minimum width=1em, yshift=-0.5em, red] (lbl6) {\footnotesize{\emph{No}}};
        \node [plain, right=2.5em of lbl6, minimum height=1em, minimum width=1em, yshift=1.5em, xshift=-1.5em, red] (lbl7) {\footnotesize{\emph{No}}};
        \node [plain, above=1.8em of clean, minimum height=1em, minimum width=1em, xshift=-0.1em, red] (lbl9) {\footnotesize{\emph{Yes}}};
        \node [plain, above=1.8em of chaintad, minimum height=1em, minimum width=1em, yshift=0.5em, xshift=-1.5em, red] (lbl10) {\footnotesize{\emph{Yes}}};
        \node [plain, above=0.3em of lbl10, minimum height=1em, minimum width=1em] (lbl11) {\footnotesize{\emph{Return Data}}};
        \node [plain, left=9.7em of lbl11, minimum height=1em, minimum width=1em] (lbl12) {\footnotesize{\emph{Return Data}}};
        \node [plain,right =2.5em of lbl11, minimum height=1em, minimum width=1em] (lbl13) {\footnotesize{\emph{Return Data}}};

        % Row Buffer
        \node [rectblock, thick, above = 2.4cm of act, minimum width=6.5em] (data) {\begin{varwidth}{4cm} \centering{\footnotesize{DATA}} \end{varwidth}};
        \node [rectblock, thick, right = 0cm of data, minimum width=4em] (tag) {\begin{varwidth}{4cm} \centering{\footnotesize{TAG}} \end{varwidth}};
        \node [rectblock, thick, right = 0cm of tag, minimum width=1pt] (isgpu) {\begin{varwidth}{4cm} \centering{\scriptsize{}} \end{varwidth}};
        \node [rectblock, thick, right = 0cm of isgpu] (chain) {\begin{varwidth}{4cm} \centering{\scriptsize{}} \end{varwidth}};
        \node [rectblock, thick, right = 0cm of chain] (reversechain) {\begin{varwidth}{4cm} \centering{\scriptsize{}} \end{varwidth}};
        \node [rectblock, thick, right = 0cm of reversechain, minimum width=1pt] (chaindirty) {\begin{varwidth}{4cm} \centering{\scriptsize{}} \end{varwidth}};
        \node [rectblock, thick, right = 0cm of chaindirty, minimum width=9em] (tad2) {\begin{varwidth}{4cm} \centering{\scriptsize{\textbf{TAD 2}}} \end{varwidth}};
        \node [rectblock, thick, right = 0cm of tad2, minimum width=8em] (row) {\begin{varwidth}{4cm} \centering{\textbf{...}} \end{varwidth}};
        \node [rectblock, thick, right = 0cm of row, minimum width=9em] (tad15) {\begin{varwidth}{4cm} \centering{\scriptsize{\textbf{TAD 15}}} \end{varwidth}};
        \node [rectblock, thick, right = 0cm of tad15, minimum width=1.2em] (cg) {\begin{varwidth}{4cm} \centering{} \end{varwidth}};
        \node [rectblock, thick, right = 0cm of cg, minimum width=2em, pattern=north west lines] (dc) {\begin{varwidth}{4cm} \centering{} \end{varwidth}};
        \draw [decorate,decoration={brace,amplitude=5pt},yshift=8.7em, xshift=-2.8em] (0,0.3) -- (4.6,0.3) node [black,midway,yshift=0.5em] {\footnotesize \textbf{TAD 1} ($128B + 8B$)};
        \draw [decorate,decoration={brace,amplitude=5pt},yshift=8.7em, xshift=3.2em] (11.75,0.3) -- (12.9,0.3) node [black,midway,yshift=0.5em] {\footnotesize $8B$ unused};

        \node [plain, below=0.4cm of chain, minimum height=1em, minimum width=1em, xshift=-1.6em] (lbl31){\begin{varwidth}{4cm} \centering{\footnotesize{2bit Chain}} \end{varwidth}};
        \node [plain, below=0.7cm of reversechain, minimum height=1em, minimum width=1em, xshift=1em] (lbl32) {\begin{varwidth}{4cm} \centering{\footnotesize{2bit Reverse Chain}} \end{varwidth}};
        \node [plain, below=0.2cm of chaindirty, minimum height=1em, minimum width=1em, xshift=2.4em] (lbl33) {\begin{varwidth}{4cm} \centering{\footnotesize{1bit Chain Dirty}} \end{varwidth}};
        \node [plain, below=0.1cm of cg, minimum height=1em, minimum width=1em, xshift=1em] (lbl34) {\begin{varwidth}{4cm} \centering{\footnotesize{CPU/GPU bitvector(15 bits)}} \end{varwidth}};
        \node [plain, below=0.1cm of isgpu, minimum height=1em, minimum width=1em, xshift=-1.8em] (lbl35) {\begin{varwidth}{4cm} \centering{\footnotesize{CPU/GPU}} \end{varwidth}};

        \draw [<-] (chain.south) to [out=-90,in=90] ++(0,-0.5);
        \draw [<-] (reversechain.south) to [out=-90,in=90] ++(0,-0.8);
        \draw [<-] (chaindirty.south) to [out=-90,in=90] ++(0,-0.3);
        \draw [<-] (cg.south) to [out=-90,in=90] ++(0,-0.2);
        \draw [<-] (isgpu.south) to [out=-90,in=90] ++(0,-0.2);

    \end{tikzpicture}
}

\newcommand{\hsacpu}[0]{
    \begin{tikzpicture}[auto, >=latex']
        %CPU
        \node [blocksmall, minimum height=1em, minimum width=1em] (cpu1) {\begin{varwidth}{4cm}\centering{CPU\\ Core 1} \end{varwidth}};
        \node [blocksmall, above = 0.1cm of cpu1, minimum height=1em, minimum width=1em] (l11){\begin{varwidth}{4cm} \centering{L1} \end{varwidth}};
        
        %\node [blocksmall, right = 0.1cm of cpu1, minimum height=1em, minimum width=1em] (cpu2) {\begin{varwidth}{4cm} \centering{CPU\\ Core 2} \end{varwidth}};
        %\node [blocksmall, above = 0.1cm of cpu2, minimum height=1em, minimum width=1em] (l12){\begin{varwidth}{4cm} \centering{L1} \end{varwidth}};
        
        \node [smallcircle, right=0.15cm of cpu1] (dot1) {};
        \node [smallcircle, right=0.05cm of dot1] (dot2) {};
        \node [smallcircle, right=0.05cm of dot2] (dot3) {};
        
        %\node [blocksmall, right = 0.1cm of cpu2, minimum height=1em] (cpu3){\begin{varwidth}{4cm} \centering{CPU\\ Core 3} \end{varwidth}};
        %\node [blocksmall, above = 0.1cm of cpu3, minimum height=1em] (l13){\begin{varwidth}{4cm} \centering{L1} \end{varwidth}};
        
        \node [blocksmall, right = 0.15cm of dot3, minimum height=1em, minimum width=1em] (cpu4){\begin{varwidth}{4cm} \centering{CPU\\ Core N} \end{varwidth}};
        \node [blocksmall, above = 0.1cm of cpu4, minimum height=1em, minimum width=1em] (l14){\begin{varwidth}{4cm} \centering{L1} \end{varwidth}};
        \node [blocksmall, above = 2.01cm of dot3, minimum height=3em, minimum width=6em] (l24){\begin{varwidth}{4cm} \centering{L2} \end{varwidth}};
        
       % \begin{scope}[on background layer]
            %\node[group, inner sep = 3pt, line width=0.3mm, draw=red, yshift=1em, fit=(cpu1)(cpu2)(cpu3)(cpu4)(l11)(l12)(l13)(l14)(l24)] (grpcpu) {};
       % \end{scope}
        
        
        %GPU
        \node [blocksmall, right = 0.6cm of cpu4, minimum height=1.5em, minimum width=1em] (sm1) {CU};
        \node [blocksmall, above = 0.1cm of sm1, minimum height=1pt, minimum width=1em] (sm1l1) {\scriptsize{L1}};
        \node [blocksmall, right = 0.15cm of sm1, minimum height=1.5em, minimum width=1em] (sm2) {CU};
        \node [blocksmall, above = 0.1cm of sm2, minimum height=1em, minimum width=1em] (sm2l1) {\scriptsize{L1}};
        \node [blocksmall, right = 0.15cm of sm2, minimum height=1.5em, minimum width=1em] (sm3) {CU};
        \node [blocksmall, above = 0.1cm of sm3, minimum height=1em, minimum width=1em] (sm3l1) {\scriptsize{L1}};
        \node [blocksmall, right = 0.15cm of sm3, minimum height=1.5em, minimum width=1em] (sm4) {CU};
        \node [blocksmall, above = 0.1cm of sm4, minimum height=1em, minimum width=1em] (sm4l1) {\scriptsize{L1}};
        \node [blocksmall, above = 2.3cm of sm1, minimum height=1.5em, minimum width=1em] (sm5) {CU};
        \node [blocksmall, below = 0.1cm of sm5, minimum height=1em, minimum width=1em] (sm5l1) {\scriptsize{L1}};
        \node [blocksmall, right = 0.15cm of sm5, minimum height=1.5em, minimum width=1em] (sm6) {CU};
        \node [blocksmall, below = 0.1cm of sm6, minimum height=1em, minimum width=1em] (sm6l1) {\scriptsize{L1}};
        \node [blocksmall, right = 0.15cm of sm6, minimum height=1.5em, minimum width=1em] (sm7) {CU};
        \node [blocksmall, below = 0.1cm of sm7, minimum height=1em, minimum width=1em] (sm7l1) {\scriptsize{L1}};
        \node [blocksmall, right = 0.15cm of sm7, minimum height=1.5em, minimum width=1em] (sm8) {CU};
        \node [blocksmall, below = 0.1cm of sm8, minimum height=1em, minimum width=1em] (sm8l1) {\scriptsize{L1}};  
        
        \node [blocksmall, minimum height=1.1em, minimum width=6em] at (5.6,1.45)
        (l2gpu) {\begin{varwidth}{4cm} \centering{\small{Coherent GPU \\L2}}\end{varwidth}};
        
         %\begin{scope}[on background layer]
            \node[group, inner sep = 3pt, line width=0.3mm, draw=red, fit=(sm1)(sm1l1)(sm2)(sm2l1)(sm3)(sm3l1)(sm4)(sm4l1)(sm5)(sm5l1)(sm6)(sm6l1)(sm7)(sm7l1)(sm8)(sm8l1)(l2gpu)] (grpgpu) {};
         %\end{scope}                        
        
        
        %Interconnect
        \node [blocksmall, minimum height=1.5em, minimum width=10.7em] at (1.5,1.5)
        (interconnect) {Cache Coherent Interconnect};
                
        %Connections to interconnect
        \draw [-,very thick] (l11.north) to ++(0, 0.15); 
        \draw [-, very thick] (l24.south) to ++(0, -0.3);
        \draw [-, very thick] (l2gpu.west) to ++(-0.82,0);
        \draw [-, very thick] (l14.north) to ++(0, 0.15);
        %\draw [-, very thick] (l13.north) to ++(0, 0.15);
        %\draw [-, very thick] (l12.north) to ++(0, 0.15);
        
        % Memory Controller
        \node [blocksmall, minimum height=1em, minimum width=9em, below=0.1cm of cpu4, xshift=-2em] (mc1){\begin{varwidth}{4cm} \centering{Memory Controller} \end{varwidth}};
        \node [blocksmall, minimum height=1em, minimum width=9em, right=0.05cm of mc1] (mc2){\begin{varwidth}{4cm} \centering{Memory Controller} \end{varwidth}};  
        
        %\draw [-, very thick] (-0.78,-0.6) to (-0.78, 1.5) to (interconnect.west);
        \draw [-, very thick] (2.8,-0.6) to (2.8, 1.25);
        \draw [-, very thick] (3.07,-0.6) to (3.07, 1.25);
        
        %Chip
        \begin{scope}[on background layer]
             \node[group, inner sep = 4pt, line width=0.6mm, fill=cyan!70, fit=(cpu1)(cpu4)(l11)(l14)(l24)(interconnect)(mc1)(mc2)(grpgpu)] (grpchip) {};
        \end{scope}

        % GPU L1 to L2 connections
        \draw [-,very thick] (sm1l1.north east) -- (l2gpu);
        \draw [-,very thick] (sm2l1.north) -- (l2gpu);
        \draw [-,very thick] (sm3l1.north) -- (l2gpu);
        \draw [-,very thick] (sm4l1.north) -- (l2gpu);
        \draw [-,very thick] (sm5l1.south east) -- (l2gpu);
        \draw [-,very thick] (sm6l1.south) -- (l2gpu);
        \draw [-,very thick] (sm7l1.south) -- (l2gpu);                
        \draw [-,very thick] (sm8l1.south) -- (l2gpu);

        % Off package Region
        \node [blocksmall, below = 3em of mc1, draw=black!30, fill=black!30, minimum height=0.5em, minimum width=7em] (mm1){};
%        \node [blocksmall, draw=blue, right = 0.2cm of mm1, minimum height=1.2em, minimum width=1.2em] (mm2){};
 %       \node [blocksmall, draw=blue, right = 0.2cm of mm2, minimum height=1.2em, minimum width=1.2em] (mm3){};
  %      \node [blocksmall, draw=blue, right = 0.2cm of mm3, minimum height=1.2em, minimum width=1.2em] (mm4){};
        
        \node [blocksmall, draw=black!30, fill=black!30, minimum height=0.5em, right = 0.8cm of mm1, minimum width=7em] (mm7){};
%        \node [blocksmall, draw=blue, right = 0.2cm of mm7, minimum height=1.2em, minimum width=1.2em] (mm8){};
%        \node [blocksmall, draw=blue, right = 0.2cm of mm8, minimum height=1.2em, minimum width=1.2em] (mm9){};
%        \node [blocksmall, draw=blue, right = 0.2cm of mm9, minimum height=1.2em, minimum width=1.2em] (mm10){};
        
        \node [blocksmall, draw=white, xshift=4em, below = 0.5cm of mm7, minimum height=1em, minimum width=1.2em] (lbl1){DIMMs};
        \node [blocksmall, draw=white, yshift=-4em, xshift=-13em, above = 0.1cm of mm7.south east, minimum height=1em, minimum width=1.2em] (lbl2){Off Package region};
        
        
        % Memory Grouping
        \begin{scope}[on background layer]
              \node[group, inner sep = 1pt, line width=0.6mm, fill=black!30, fit=(mm1)] (mmgrp1) {};
        \end{scope}
        \begin{scope}[on background layer]
              \node[group, inner sep = 1pt, line width=0.6mm, fill=black!30, fit=(mm7)] (mmgrp2) {};
        \end{scope}
        \node[group, inner sep = 3pt, line width=0.3mm, fit=(mmgrp1)(mmgrp2)(lbl1)(lbl2)] (mmgrp3) {};
        
        \draw [-,very thick, xshift=-2em] (1.2,-2.7) to (1.2, -3) to (3,-3) to (3,-2.7);
        \draw [-,very thick, xshift=7em,] (1.2,-2.7) to (1.2, -3) to (3,-3) to (3,-2.7);
%        \draw [-,very thick] (1.5,-2.15) to (1.5, -2.35) to (4.9,-2.35) to (4.9,-2.15);

        % Big Arrows
        \node[double arrow, draw, right = 0.1mm of grpchip.south, yshift=1em,
        rotate=-90, minimum height=3em, minimum width=1.3em, line width=0.5mm, fill=white](arrow2){};

    \end{tikzpicture}
}

\newcommand{\stackdram}[0]{
    \begin{tikzpicture}[auto, >=latex']
        %DRAM Cache  
        \node [block, draw=red, dashed, minimum height=7em, minimum width=4em] at (0,6.2) (sid){\begin{varwidth}{4cm} \centering{\footnotesize{Silicon\\ interposed\\ DRAM}} \end{varwidth}};
        
        \node [block, draw=red, right = 2.1cm of sid, dashed, minimum height=7em, minimum width=4em] (sid1){\begin{varwidth}{4cm} \centering{\footnotesize{Silicon\\ interposed\\ DRAM}} \end{varwidth}};
        
        \node [block, draw=red, minimum height=0.1pt, rounded corners=1pt, inner sep=1pt, minimum width=16em] at (1.77, 4.8) (blk1){\begin{varwidth}{4cm} \end{varwidth}};
        
        \node [block, draw=black, above = 0.05cm of blk1, minimum height=0.1em, minimum width=1em] (cpud){\begin{varwidth}{4cm} \centering{\scriptsize{CPU-GPU Die}} \end{varwidth}};
        
        \node [block, draw=red, above = 0.2cm of cpud, minimum height=4em, dashed, minimum width=4em] (vsd) {\begin{varwidth}{4cm} \centering{\footnotesize{Vertically\\ Stacked\\ DRAM\\}} \end{varwidth}};
        
        \node [blocksmall, draw=white, above = 0.01cm of sid.north east, minimum height=1em, minimum width=0.2em] (lbl3){\footnotesize{On Package region}};
        
        \node[group, inner sep=7pt, rounded corners, fit=(sid)(vsd)(sid1)(cpud)(blk1)(lbl3)] (g1) {};
        
        % TSV        
        \draw [-,very thick] (1.35,5.38) to (1.35,5.6);
        \draw [-,very thick] (1.65,5.38) to (1.65,5.6);
        \draw [-,very thick] (1.95,5.38) to (1.95,5.6);
        \draw [-,very thick] (2.25,5.38) to (2.25,5.6);
        
        % semicircles
        \node [smallcircle] at (-0.6,4.9) (s4) {};
        \node [smallcircle] at (-0.3,4.9) (s1) {};
        \node [smallcircle] at (0,4.9) (s2) {};
        \node [smallcircle] at (0.3,4.9) (s3) {};
        \node [smallcircle] at (0.6,4.9) (s4) {};
        %\node [smallcircle] at (0.7,4.9) (s3) {};
        %\node [smallcircle] at (1.0,4.9) (s4) {};

        \node [smallcircle] at (1,4.9) (s5) {};
        \node [smallcircle] at (1.3,4.9) (s6) {};
        \node [smallcircle] at (1.6,4.9) (s7) {};
        \node [smallcircle] at (1.9,4.9) (s8) {};
        \node [smallcircle] at (2.2,4.9) (s9) {};
        \node [smallcircle] at (2.5,4.9) (s10) {};

        \node [smallcircle] at (3.1,4.9) (s11) {};
        \node [smallcircle] at (3.4,4.9) (s12) {};
        \node [smallcircle] at (3.7,4.9) (s13) {};
        \node [smallcircle] at (4,4.9) (s14) {};
        \node [smallcircle] at (4.3,4.9) (s15) {};
        
\iffalse        
        % Off package Region
        \node [blocksmall, draw=black!30, fill=black!30, minimum height=0.5em, minimum width=0.5em] at (5.65,5.15)(mm1){};
        \node [blocksmall, draw=black!30, fill=black!30, above = 0.15cm of mm1, minimum height=0.5em, minimum width=0.5em] (mm2){};
        \node [blocksmall, draw=black!30, fill=black!30, above = 0.15cm of mm2, minimum height=0.5em, minimum width=0.5em] (mm3){};
        \node [blocksmall, draw=black!30, fill=black!30, above = 0.15cm of mm3, minimum height=0.5em, minimum width=0.5em] (mm4){};
        \node [blocksmall, draw=black!30, fill=black!30, above = 0.15cm of mm4, minimum height=0.5em, minimum width=0.5em] (mm5){};
        %\node [blocksmall, draw=blue, above = 0.2cm of mm5, minimum height=0.5em, minimum width=0.5em] (mm6){};

        \node [blocksmall, draw=black!30, fill=black!30, minimum height=0.5em, right = 0.55cm of mm2, minimum width=0.5em] (mm7){};
        \node [blocksmall, draw=black!30, fill=black!30, above = 0.15cm of mm7, minimum height=0.5em, minimum width=0.5em] (mm8){};
        \node [blocksmall, draw=black!30, fill=black!30, above = 0.15cm of mm8, minimum height=0.5em, minimum width=0.5em] (mm9){};
        \node [blocksmall, draw=black!30, fill=black!30, above = 0.15cm of mm9, minimum height=0.5em, minimum width=0.5em] (mm10){};
        \node [blocksmall, draw=black!30, fill=black!30, above = 0.15cm of mm10, minimum height=0.5em, minimum width=0.5em] (mm11){};
       % \node [blocksmall, draw=blue, above = 0.2cm of mm11, minimum height=0.5em, minimum width=0.5em] (mm12){};
        \begin{scope}[on background layer]
            \node [blocksmall, draw=white, below = 0.3cm of mm7, minimum height=1em, minimum width=1.2em] (lbl1){\scriptsize{DIMMs}};
        \end{scope}
        \begin{scope}[on background layer]
            \node [blocksmall, draw=white, above = 0.21cm of mm5.north east, minimum height=0.1em, minimum width=0.1em] (lbl2){\begin{varwidth}{4cm} \scriptsize{Off Package\\region}\end{varwidth}};
       \end{scope}
        
        % Memory Grouping
        \begin{scope}[on background layer]
              \node[group, minimum width=0.1em, minimum height=0.1em, fill=black!30, fit=(mm1)(mm2)(mm3)(mm4)(mm5)] (mmgrp1) {};
        \end{scope}
        \begin{scope}[on background layer]
              \node[group, minimum width=0.1em, minimum height=0.1em, fill=black!30, fit=(mm7)(mm8)(mm9)(mm10)(mm11)] (mmgrp2) {};
        \end{scope}
        \node[group, inner sep = 3pt, rounded corners, line width=0.3mm, fit=(mmgrp1)(mmgrp2)(lbl1)(lbl2)] (mmgrp3) {};
        %\node[group, inner sep = 3pt, line width=0.3mm, fit=(mmgrp1)(mmgrp2)] (mmgrp3) {};
        
        %\draw [-,very thick] (7.9,6.5) to (7.9, 6.3) to (11.3,6.3) to (11.3,6.5);
        %\draw [-,very thick] (8.5,5.35) to (8.5, 5.15) to (11.9,5.15) to (11.9,5.35);
        \draw [-,very thick] (5.9, 5.1) to (6.05, 5.1) to (6.05, 7.0) to (5.9, 7.0);
        \draw [-,very thick] (6.73, 5.55) to (6.87, 5.55) to (6.87, 7.4) to (6.73, 7.4);
        
        % Big Arrows
        \node[single arrow, draw, below = 0.1mm of cpud.south, xshift=1em,
        yshift=-1.8em, rotate=-90, minimum height=4.24em, minimum width=3em, line width=0.5mm, fill=white](arrow1){};
        \node[double arrow, draw, right = 0.1mm of sid1.east,
        yshift=0em, rotate=0, minimum height=3em, minimum width=1.3em, line width=0.5mm, fill=white](arrow2){};
\fi        
        % Labels
        \node [blocksmall, opacity=0,text opacity=1, draw=white, minimum height=1em, minimum width=1.2em] at (0.4,4.0)
        (lbl4){\begin{varwidth}{4cm} \centering{\scriptsize{Through Silicon\\Vias}}\end{varwidth}};
        \node [blocksmall, draw=white, right=1.2cm of lbl4, minimum height=1em, minimum width=1.2em] (lbl5) {\begin{varwidth}{4cm} \centering{\scriptsize{Silicon\\Interposer}}\end{varwidth}};
        %\node [blocksmall, draw=white, right=0.1cm of lbl5, minimum height=1em, minimum width=1.2em, yshift=0.4em, inner sep=2pt] (lbl6) {\begin{varwidth}{4cm} \centering{\scriptsize{External Memory \\Interface (e.g.DDR)}} \end{varwidth}};
        
        % Label Arrows
        \draw[->, very thick] (lbl5.north) to (blk1);
%        \draw[<-, very thick] (4.8, 5.8) to [in=90,out=-90] ++(0,-1.3);
        \draw[<-, very thick] (1.32,5.49) to [in=0,out=180] ++(-0.7,0) to [in=90,out=-90] ++(0,-1.12);
        
    \end{tikzpicture}
}
